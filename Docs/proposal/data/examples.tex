\section{SMURF Examples}

\subsection{Simple Example}
The simple SMURF program below first defines the pitch classes for the prime row in line 1. 
Using the prime row as a base, the program creates a score with a single measure using the 
notes generated by transposing the prime row by three semitones (line 2), inverting the 
resulting transposed row (line 3), and then reversing the inverted row from the prior step 
(line 4). Lines 2-4 use the \emph{trans}, \emph{inver}, and \emph{rev} library routines to apply 
transposition, inversion, and reversal to a list, respectively. Line 5 invokes a library 
routine called \emph{rowToNotes} that converts each pitch class in a tone row to a list of 
notes given beat and register value mappings for each pitch class in the tone row. For 
simplicity, all the notes in this example are quarter notes (4) in register 0. Line 6 defines 
a measure using the first two notes in the list of notes returned in line 5; the second half 
of the measure consist of two rest quarter notes in register 0. A rest is defined as a note 
with pitch class -1.  Line 7 uses the keyword \textbf{genScore} along with the measure defined 
in the prior line as an argument. The compiler translates the keyword \textbf{genScore} down 
to C+OpenGL code to create a score sheet based on the given list of measures.     

\begin{verbatim}
1: prime = [0,2,4,6,8,10]
2: p3    = trans 3 prime
3: i3    = inver p3
4: ri3   = rev i3
5: firstNotes = rowToNotes ri3 [4,4,4,4,4,4] [0,0,0,0,0,0]
6: firstMeasure = head firstNotes:(head (tail firstNotes)):(-1,4,0):(-1,4,0):[]
7: genScore [firstMeasure]
\end{verbatim}

\subsection{Interesting Example}

TBD
