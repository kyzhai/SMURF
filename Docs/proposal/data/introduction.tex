\section{Background: What is Serialism?}

In general, serialism is a musical composition technique where a set of values, chosen through some methodical process, 
generates a sequence of musical elements. Its origins are often attributed to Arnold Schoenberg's twelve-tone technique, which
he began to use in the 1920s. In this system, each note in the chromatic scale is assigned an integer value, giving us a set of twelve
"pitch classes" (see fig. 1).
A composer utilizing this method then takes each of these integers, and orders them into a $twelve$ $tone$ $row$, where 
each number appears exactly once. We refer to this row as the $prime form$ of a piece, and conventionally refer 
to it as $P_0$. 

The composer can then generate other rows that are derived from $P_0$ through three types of transformations:
transposition, inversion, and retrograde. In each of these transformations, we always use mod 12 arithmetic to preserve the 
numbering system of our pitch classes. Transposing a row consists of taking each pitch class in the row and adding the same number 
to each. If we transpose $P_0$ by four semitones, we add four mod twelve to each pitch class in $P_0$ and end up with a new row 
called $P_4$. In general, $P_x$ is a transposition of $P_0$ by $x$ semitones. To invert a row, we flip each interval between two 
pitch classes in that row. For example, if $P_0$ starts with pitch classes 0-4-1, then we have an interval of +4 between the first two 
pitches and -3 between the second two. In the inverse of $P_0$ (called $I_0$), the first interval would be -4 and the second would 
be +3, giving us 0-8-11 as our first three pitch classes. The subscript of $I_x$  refers both to the number of transpositions required 
to arrive at $I_x$ from $I_0$, and to the prime row $P_x$ that would need to be inverted to generate $I_x$. The final row 
operation is a retrograde transformation, which merely consists of reading a row backwards. That is, $R_x$ is generated by reading 
the pitch classes of $P_x$ in their opposite order. One can also have a retrograde inversion; $RI_x$ is generated by reading the 
pitch classes of $I_x$ backwards.

Once a composer chooses a $P_0$, the three transformations outlined above can be applied to varying degrees to generate a $twelve$ $tone$ $matrix$, which will contain each $P$ row as a row in the matrix and each $I$ row as a column.
Furthermore, all of the $R$ and $RI$ rows are found by reading the rows in the matrix from right to left or the columns 
from bottom to top, respectively. An example of a 
twelve tone matrix from one of Shoenberg's pieces can be found below (fig. 2). Finally, using the twelve tone matrix as a guide,
the composer picks various rows and columns to serve as melodic and harmonic elements in their composition, resulting in a piece
of serial music.
%Possible transformations
%Notation





