\section{Motivation}

Our group decided to create a language that assists a composer in creating music 
using the twelve tone method described above. We plan on making this language 
functional and compile into C. The compilation process will create a C program 
that will use openGL to create an image representing the music created. 

Twelve tone serialism is a mathematically intensive method of creating music which 
involves mapping notes to numbers. It is natural to work with twelve tone rows 
using a programming language since the method treats notes like numbers that 
can be added and subtracted. We plan on our language making twelve tone compilation 
easier using data types and functions specifically for the purpose of creating music in 
this way. By simplifying the method of inverting and transposing rows composers can focus 
more on how to exploit new ways to make music in this fashion and worry less about 
creating matrices. 

We chose to implement a functional language because of the clear and 
succinct programs that functional languages produce. In addition, the well known ability
of functional languages to work on lists is advantagous for twelve tone serialism because
most arithematic for its compositions is done in rows and columns. As a group we are also 
interested on how a functional language compiler works. 

Instead of compiling our language into byte code where notes are interpreted and 
played, we think it is more interesting to create a language that is compiled 
into a score. Scores are made up of lines and dots, which are easy to create in 
a graphics library like openGL. We decided to compile our language into C since it is a 
significantly less abstract language that has access to openGL library calls. One benefit 
of compiling into C instead of a language to be interpreted as music is that programs in our 
language can create functions that can be used in other programs to do the same 
transformation of rows. That way our language is not limited to functions defined in 
our libraries or what is in the current file. 

Overall we hope to use the simplicity of a functional language to help composers write 
complex, new, and interesting music based on twelve tone serialism. These new compositions 
can then be printed and used by musicians to play. This simplifies the composer's task of 
converting computer generated music to one in which musicians can easily read. 
