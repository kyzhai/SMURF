\section{Motivation}

Our group has decided to create a language that will assist a composer in creating music 
using the twelve tone method described above. We plan on making this language 
functional and compile into C. The compilation process will create a C program 
that will use openGL to create an image representing the music created. 

Twelve tone serialism is a mathematically intensive method of creating music which 
involves mapping notes to numbers. It is very natural to work with twelve tone rows 
using a programming language since the method just treats notes like numbers that 
can be added and subtracted from. We plan on our language making twelve tone compilation 
easier using data types and functions specifically for the purpose of creating music in 
this way. By simplifying the method of inverting and transposing rows composers can focus 
more on how to exploit new ways to make music in this fashion and worry less about 
creating matrices. 

We chose to implement our language as a functional language because of the clear and 
succinct programs that functional languages produce. It also makes sense for our language 
to be functional because functional languages also are well known for their ability to 
work on lists and most arithmetic done in twelve tone serialism is done on rows or 
columns. As a group we are also very interested on how a functional language compiler 
works. 

Instead of compiling our language into byte code which can be interpreted and play the 
notes we thought it would be more interesting to create a language that could be compiled 
into a score. Scores are made up of just lines and dots and would be easy to create in 
a graphics library like openGL. We decided to compile our language into C since it was a 
significantly less abstract language which had access to openGL library calls. One benefit 
of compiling into C and not a language to be interpreted as music is that programs in our 
language could be created that do not actually output any music but instead create 
functions that could be used in other programs to do the same transformation of rows. 
That way our language is not just limited to functions defined in our libraries or 
what is in the current file. 

Overall we hope to use the simplicity of a functional language to help composers write 
complex, new, and interesting music based on twelve tone serialism. These new compositions 
would then be able to be printed and handed to musicians to play. This simplifies the 
composers task of converting music in computer format to one which musicians will be 
able to read easily. 
