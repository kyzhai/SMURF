\section{Motivation}

Twelve tone serialism is a mathematically intensive method of creating music which 
involves mapping notes to numbers. It is natural to work with twelve tone rows 
using a programming language since the method treats notes like numbers that 
can be added and subtracted. Our language will make twelve tone composition 
easier by using data types and functions that cater to the needs of a serial composer.
By simplifying the method of inverting and transposing rows, composers can focus 
more on how to exploit new ways to make serial music and worry less about 
creating matrices. 

We chose to implement a functional language because of the clear and 
succinct programs that functional languages produce. In addition, the well known ability
of functional languages to work on lists is advantagous for twelve tone serialism, because
most serial arithmetic operations use rows and columns from the twelve tone matrix as operands. 
As a group we are also interested on how a functional language compiler works. 

Instead of compiling our language into byte code where notes are interpreted and 
played, we think it is more interesting to create a language that, once compiled, 
generates a score. Scores are made up of lines and dots, which are easy to create in 
a graphics library like openGL. We decided to compile our language into C since it is a 
significantly less abstract language that also has access to openGL library calls. Compiling into C
 (as opposed to compiling to a language that generates an audio file) is advantageous in 
that programs in our language can create functions that can be used in other programs to do the same 
transformation of rows. That way our language is not limited to functions defined in 
our libraries or what is in the current file. 

Overall we hope to use the simplicity of a functional language to help composers write 
complex, new, and interesting music based on twelve tone serialism. These new compositions 
can then be printed and used by musicians for performances. This simplifies the composer's task of 
converting computer generated music to a form that musicians can easily read. 
