\section{Motivation}

Identify the targets you want to study.  Define their observable characteristics
by giving their identifiers (e.g. common name if any, NGC or HD or other catalog
entry you can find with AstroCC and Simbad), celestial coordinates, optical
magnitudes, and angular size if the object is extended, that is, non-stellar.
Please select targets that we have a good likelihood of observing:
\begin{itemize}
\item Not fainter than magnitude 19 (18 is better)
\item Not larger than $0.5^\circ$, but see below
\item Above the horizon at either observatory for several hours this fall
\end{itemize}

A single 100 second exposure with the 0.5 meter telescopes will reach magnitude
18 on a clear night.  Accurate quantitative measurements require a little
brighter, or longer total accumulated exposures.  The telescopes resolve 1
arcsecond in two pixels and have a field of view of $0.6^\circ$.  Larger fields
must be mosaics of several exposures. These factors will affect your
choices.  

For planetary imaging the CDK20's can take exposures as short as 0.01 seconds. 
The longest practical single exposure is about 300 seconds, but typically we
take 100 second exposures and add them in order to make small guiding
corrections between exposures. Use AstroCC with Stellarium to assure that the
targets are observable this season.

