\section{Syntax}

Assuming that the best telescope for your work is one of the two 0.5 meters
(CDK20N at Moore Observatory, CDK20S at Mt. Kent), you will have a choice of
filters:  Sloan filter set (g, r, i, or z),  Johnson-Cousins (U, B, V, R, or I),
color imaging (B, G, R, or clear), and narrow band (S $[II]$, red continuum,
H$\alpha$,  O$[III]$.  Identify which filters are of interest.

A typical exposure time for a magnitude 12 star to about half saturation is 100
seconds, but it depends on the filter choice.  Based on this, estimate how many
exposures you will need, and what total time you require.  In some cases, for
example studying an eclipsing or variable star, or an exoplanet transit, you
would use only one filter and make many measurements over a night.  In others,
you may make only a few exposures in each filter, and try many different
filters.   Changing filter sets takes an operator and several minutes, but
changing filters within one set (e.g. a different Sloan filter) takes only a few
seconds.

We have other telescopes that may be available at Moore Observatory this season.
There is a wide field astrograph that has a field of view of $4^\circ$ and is a
fast $f/4$,  especially good for large nebula, comets, or surveys.  A 14-inch
(0.36 meter) Celestron  telescope can be equipped with a fast camera for
planetary imaging.  A 27-inch (0.7 meter)  corrected Dall-Kirkham is scheduled
to be be delivered to Australia this fall, although we are unsure of the actual
date it could see light yet.  

