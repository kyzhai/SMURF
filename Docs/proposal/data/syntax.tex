\section{Syntax}

SMURF is a functional programming language loosely modeled off of Haskell. Functions are the main building 
blocks of SMURF. It has immutable memory, no global variables and no I/O. A typical program consists of
constant definitions and functions, each separated by new line characters.

\subsection{Types}

\subsubsection{Atomic Types}
\begin{itemize}
\item Integer: int
\item Boolean: bool
\end{itemize}

\subsubsection{Structured Types}
Structured types hold groups of elements
\begin{itemize}
\item Tuples: (a, ..., n), where items a - n are elements in the tuple
  \begin{itemize}
  \item Elements can have different types 
  \end{itemize}
\item Lists:  [a, ..., n], where items a - n are elements in the list 
  \begin{itemize}
  \item Elements must have same type
  \end{itemize}
\end{itemize}

\subsubsection{Beat}
A Beat represents a length of musical time. It has a Time tag and integer type. 
\begin{itemize}
  \item Must have the string ``Time" followed by an integer that is a power of 2 and \textless\space 32 in declaration
  \begin{itemize}
    \item whole note: Time 1
    \item half note: Time 2
    \item quarter note: Time 4
    \item eighth note: Time 8
    \item sixteenth note: Time 16
    \item thirty-second note: Time 32
  \end{itemize}
  \item Uses + operator to combine Time but only adds two operands that contain the same integer; recursively 
  checks for Time operands that contain the same integers until only unequal Time integers are left
  \begin{itemize}
    \item Time 4 + Time 16 + Time 16 + Time 16 + Time 16
    \item Time 4 + Time 8 + Time 16 + Time 16
    \item Time 4 + Time 8 + Time 8 
    \item Time 4 + Time 4 = Time 2 (quarter note + quarter note = half note)
  \end{itemize}
\end{itemize}

\subsubsection{Note}
A Note is a tuple of three integers and is declared as (pc: int, beat: Beat, register: int)
\begin{itemize}
\item pc (pitch class): represented by integers 0-11
  \begin{itemize}
  \item A Note with pc = -1 represents a rest. In this special case, the register for the Note only matters in relation
  to whether the rest lies on the treble or bass clef (i.e. whether the register is positive or negative)
  \end{itemize}
\item beat: has type Beat
\item register: represented by integers -2 to 1
  \begin{itemize}
  \item \begin{music}  \trebleclef  \end{music}  Treble Clef: notes middle C and higher represented by 0 and 1  
    \begin{itemize}
    \item middle C to the first B above middle C: 0
    \item first C above middle C to next highest B: 1
    \end{itemize}
  \item \begin{music} \bassclef  \end{music}  Bass Clef: notes lower than middle C represented by -1 and -2
    \begin{itemize}
    \item B directly below middle C to first C below middle C: -1
    \item next lowest B to next lowest C: -2
    \end{itemize}
  \end{itemize}
\end{itemize}

\subsubsection{Chord}
A Chord is a list of notes and is declared as [Note]. The compiler will check that all notes in the list have 
the same beat count.

\subsubsection{Measure}
A Measure is a list of chords and is declared as [Chord]. The compiler will check that every 
measure is composed of 4 beats, where a quarter note constitutes one beat
(we are assuming that musical scores will be in \Takt{4}{4} time). A list of measures is passed to the 
keyword genScore, which generates and outputs an image of the musical score.

\subsection{Functions}
Functions are the main building blocks of the language.
\begin{itemize}
\item Function declarations must declare type (can declare general type)
\item Function declarations must be on own line
\item All functions are first order
\item No explicit return statement
\item Pattern matching, guards, and if-then-else clauses used
  \begin{itemize}
  \item Each pattern matching pattern must be on own line
  \item Guards and if-then-else clauses do not have newline restrictions
  \end{itemize}
\end{itemize}

\subsection{Variables}
Variables can be declared globally or locally. A global variable is declared with an assignment
operator and acts as a constant e.g. x = 4. A local variable is declared with a let ... in ...
expression. Multiple bindings may be declared in the let expression, and the bindings are
mutually recursive. The bindings only exist in the expression following the let expression. In the
example that follows, x is bound to 3 and y is bound to 4, and these bindings only exist in
the expression following the let ... in ... expression.

\begin{verbatim}
let x = 3
    y = 4
in x + y
\end{verbatim}

\subsection{Operators}

  \subsubsection{Comment Operators}
  SMURF allows nested, multiline comments in addition to single line comments.
    \begin{table} [h]
	\centering
    \begin{tabular}{lll}
    \hline\hline
    Operator & Description & Example \\
    \hline\hline
      /* */ & Multiline comments, nesting allowed & /* This /* is all */ commented */ \\ \hline
      // & Single-line comment & // This is a comment \\ \hline
    \end{tabular}
  \end{table}

  \subsubsection{Arithmetic Operators}
  SMURF allows assignment and addition, subtraction, and modulus on expressions that evaluate
	to integers. These operators are all infix. The modulus operator
	ignores negatives e.g. \begin{verbatim} 13 % 12 is equal to -13 % 12 \end{verbatim}.
  \begin{table} [h]
	\centering
    \begin{tabular}{lll}
    \hline\hline
    Operator & Description & Example \\
    \hline\hline
      = & Assignment operator & a = 4 \\ \hline
      + & Integer arithmetic: plus  & a + 2 \\ \hline
      $-$ & Integer arithmetic: minus  & 5 $-$ a \\ \hline 
      \% & Integer arithmetic: modulus, ignores negatives  & 14 mod 12 \\ \hline
    \end{tabular}
  \end{table}

  \subsubsection{Comparison Operators}
  SMURF allows comparison operations between expressions that evaluate to integers.
    \begin{table} [h]
	\centering
    \begin{tabular}{lll}
    \hline\hline
    Operator & Description & Example \\
    \hline\hline
      \textless  & Less than & if a \textless\space  5 then True else False \\ \hline
      \textgreater  & Greater than & if a \textgreater\space  5 then True else False  \\ \hline
      \textless=  & Less than or equal to & if a \textless= 5 then True else False \\ \hline
      \textgreater= & Greater than or equal to & if a \textgreater= 5 then True else False \\ \hline
    \end{tabular}
  \end{table}
  
  \subsubsection{Boolean Operators}
  SMURF allows logical negation, conjunction, and disjunction, in addition to structural comparison and boolean 
  notation for use with guards.
    \begin{table} [h]
	\centering
    \begin{tabular}{lll}
    \hline\hline
    Operator & Description & Example \\
    \hline\hline
       == & Structural comparison & if a == 5 then a = True else a = False \\ \hline
       not & Logical negation & if not a == 5 then True else False \\ \hline
       \&\& & Logical conjunction & if b \&\& c  then True else False \\ \hline
       \textbar\textbar & Logical disjunction & if b \textbar\textbar\space   c  then True else False \\ \hline
     \end{tabular}
  \end{table}
  
  \subsubsection{List Operators}
  SMURF allows concatenation and construction of lists.
    \begin{table} [h]
	\centering
    \begin{tabular}{lll}
    \hline\hline
    Operator & Description & Example \\
    \hline\hline
       ++ & Concatenation: concat & [1,2,3] ++ [4,5,6] (result is [1,2,3,4,5,6]) \\ \hline
       : & Construction: cons & 1 : [2,3,4] (result is [1,2,3,4]) \\ \hline
    \end{tabular}
  \end{table}
  
  \subsubsection{Function Operators}
  SMURF allows type, argument, and function return type specification in addition to concatenation and construction 
  operations
    \begin{table} [h]
	\centering
    \begin{tabular}{lll}
    \hline\hline
    Operator & Description & Example \\
    \hline\hline
       :: & Type specification & returnIntFunc :: Int \\ \hline
       \textendash\textgreater & Argument and function return type specification
         & isPositiveNum :: Int \textendash\textgreater\space Bool  \\ \hline
       \textbar & Boolean operator used with guards & isLowOrSeven num :: [Int] \textendash\textgreater\space Bool\\ 
         && \textbar\space (num \textless 5 \textbar\textbar\space num == 7) = True \\
         && \textbar\space otherwise = False\\ \hline
    \end{tabular}
  \end{table}

\subsection{Keywords}
\begin{table} [h]
	\centering
    \begin{tabular}{ll}
    \hline\hline
    Keywords & \\ 
    \hline\hline
      let & Specify variables and functions  \\ \hline
      in & Allow local variable binding in expression \\ \hline
      if, then, else & Specify conditional expression, else compulsory  \\ \hline
      True, False & Specify boolean logic \\ \hline
      otherwise & Specify conditional expression used with guards \\ \hline 
      genScore & Generate musical score given list of measures as argument  \\ \hline
    \end{tabular}
\end{table}

\subsection{Library Functions}
There are currently six library functions that can be used in SMURF: trans, inver, rev, head, tail, fillMeasure.

\subsubsection{trans}
Given an integer and a list (row), return a list that adds the integer to each element in the original list and 
computes mod 12 on the result of each addition. If not given an integer and then a list return an error.

\subsubsection{inver}
Given a list (row), return a list that has the original list elements inverted. If not given a list return error.

\subsubsection{rev}
Given a list (row), return a list that has the original list elements reversed. If not given a list return error.

\subsubsection{head}
Given a list of elements, return the first element in the list. If not given a list return error.

\subsubsection{tail}
Given a list of elements, return the list with the first element removed. If not given a list return error.

\subsubsection{fillMeasure}
Given a list of chords, return a boolean whether the beats of each chord in the list add up to a measure.
  \begin{itemize}
    \item Time 1 + Time 32 evaluates to False (cannot have more than a whole note in a measure)
    \item Time 4 + Time 16 evaluates to False (quarter note and sixteenth note do not contain enough beats for a measure)
    \item Time 4 + Time 4 + Time 4 + Time 4 evaluates to True (4 quarter notes in one measure)
  \end{itemize}
