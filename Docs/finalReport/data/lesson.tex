\section{Lessons Learned}

\subsection{Lindsay Neubauer}
We had a meeting at the same time every week that lasted between one and two hours that everyone attended. This time set aside to make real progress on project was crucial to our success. In the beginning of the semester we spent all the time discussing LRM-related topics and during the latter half of the semester it was split between discussion and coding. Often being in the same room, even for a short amount of time, while coding was helpful for figuring out the trickier aspects of our language. This was particularly helpful for me because OCaml was my first experience using a functional programming language and having access to others with more experience helped me pick it up quicker.
\\ \\
Another important choice we made was to designate a group leader at the beginning of this project. Our group leader was organized with tasks to discuss or complete in each meeting and helped drive the conversation in a productive way. In addition to this, we had a note taker and a person in charge of our GitHub and Latex environments. It was helpful to have �go to� people for questions and concerns that arose throughout the project.
\\ \\
After turning in our LRM we decided to divide each part of our language into slices. Each group member was in charge of a different aspect of our language and implemented that slice for each step of the compiler building process. This ownership of a part of the language was helpful and touching each step of the compiler was very helpful for learning. It also allowed each group member to work throughout the semester regardless of the progress made by others.
\\ \\
The most important learning I had from this project are understanding the functional language paradigm and knowledge on how to implement a compiler from start to finish.