\section{Tutorial}
This tutorial covers how to install, run, and write basic SMURF programs.

\subsection{Installation} 
First, untar the SMURF tarball. To compile SMURF, simply type \texttt{make} in the top level source directory. A few sample SMURF programs are located in the \textbf{examples} directory as a reference.

\subsection{Compiling and Running a SMURF Program}
A SMURF program has the extension \textbf{.sm}. To compile and run a SMURF program, execute the \texttt{toplevel.native} file as follows:\\

\texttt{\$ toplevel.native foo.sm}\\\\
A midi file containing the composition defined in your SMURF program will generate if compilation was successful. The midi file can be played using any midi compatible software such as QuickTime. Running \texttt{toplevel.native} with the -h flag will display additional options that can be supplied to \texttt{toplevel.native} when compiling a SMURF program, such as specifying an output midi file name.

\subsection{SMURF Examples}

A basic SMURF program can generate a midi file that plays a note. The following SMURF program defines a quarter note in middle C:\\

\lstinputlisting[title=simplenote.sm]{../../Code/examples/simplenote.sm}

The identifier \texttt{main} must be set in every SMURF program. In simplenote.sm, main is set to a note.  A note in SMURF consists of a pitch class or rest, the register, and the beat. In simplenote.sm, the pitch class is set to 0, the register is 2, and the 4 indicates a single beat, which turns the note into a quarter note.

Another example TBD.
