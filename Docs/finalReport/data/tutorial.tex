\section{Tutorial}
This tutorial covers how to install, run, and write basic SMURF programs.

\subsection{Installation} 
First, untar the SMURF tarball. To compile SMURF, simply type \texttt{make} in the top level source directory. A few sample SMURF programs are located in the \textbf{examples} directory as a reference.

\subsection{Compiling and Running a SMURF Program}
A SMURF program has the extension \textbf{.sm}. To compile and run a SMURF program, execute the \texttt{toplevel.native} file as follows:\\

\texttt{\$ ./toplevel.native foo.sm}\\\\
A midi file containing the composition defined in your SMURF program will generate if compilation was successful. The midi file can be played using any midi compatible software such as QuickTime. Running \texttt{toplevel.native} with the -h flag will display additional options that can be supplied to \texttt{toplevel.native} when compiling a SMURF program, such as specifying an output midi file name.

\subsection{SMURF Examples}

A basic SMURF program can generate a midi file that plays a note. The following SMURF program defines a quarter note in middle C:\\

\lstinputlisting[title=simplenote.sm]{../../Code/examples/simplenote.sm}

The identifier \texttt{main} must be set in every SMURF program. In simplenote.sm, main is set to a note.  A note in SMURF consists of a pitch class or rest, the register, and the beat. In simplenote.sm, the pitch class is set to 0, the register is 2, and the 4 indicates a single beat, which turns the note into a quarter note.

As a second example, consider the following program that plays an ascending scale followed by a series of notes interleaved with rests:\\

\lstinputlisting[title=shortcascade.sm]{../../Code/examples/shortcascade.sm}

In shortcascade.sm, \texttt{main} is set to a list of lists of chords, the latter being defined as a system in SMURF. The \texttt{makeChords} function has as input two lists of integers and a list of beats and iterates through the respective lists using recursion to generate a list of chords. The \texttt{:} operator seen in line 8 constructs a new list by appending the single note list on the left side of the operator to the list of chords. As previously mentioned, a system is a list of chords, hence \texttt{makeChords} creates a system. In line 14, a \texttt{let} expression is used to call \texttt{makeChords} providing as input the list of pitches, beats, and registers, which are defined in the declaration section of the \texttt{let} expression. Line 16 uses the concatenate operator \texttt{++} to combine two lists. On the same line, the \texttt{\$+} operator performs rhythmic addition adding together a whole note and an eighth note. The \texttt{.} operator shown in line 21 also performs rythmic addition, but adds a half of the note on the left side of the operator. In this case, the dot operator adds a quarter note and an eighth note to the half note.   This SMURF example introduces several SMURF language features, but there are additional features that are not shown in this example. 

The remainder of this document describes in more detail the SMURF language.
