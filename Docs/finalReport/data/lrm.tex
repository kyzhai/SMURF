\section{Langauge Reference Manual}

\subsection{Syntax Notation}
The syntax notation used in this manual is as follows. Syntactic 
categories are indicated by \emph{italic} type. Literal words and 
characters used in the SMURF language will be displayed in \texttt{typeset}. 
Alternatives are listed on separate lines. 

Regular expression notations are used to specify grammar patterns in this
manual.  {\it r}$*$ means the pattern {\it r} may appear zero or more times,
{\it r}$+$ means {\it r} may appear one or more times, and {\it r}$?$
means {\it r} may appear once or not at all. {\it r1}$|${\it r2} denotes an option
between two patterns, and {\it r1 r2} denotes {\it r1} followed by
{\it r2}. 

\subsection{Lexical Conventions}
SMURF programs are lexically composed of three elements: comments, tokens, and whitespace.

\subsubsection{Comments}
SMURF allows nested, multiline comments in addition to single line comments.
\begin{table} [H]
\centering
\begin{tabularx}{\textwidth}{lXl}
\hline\hline
Comment Symbols & Description & Example \\
\hline\hline
  \texttt{/* */} & Multiline comments, nesting allowed & \texttt{/* This /* is all */ commented */} \\ \hline
  \texttt{//} & Single-line comment & \texttt{// This is a comment} \\ \hline
\end{tabularx}
\end{table}


\subsubsection{Tokens}
In SMURF, a token is a string of one or more characters that is significant as a group.
SMURF has 6 kinds of tokens: {\it identifiers}, {\it keywords}, {\it constants},
      {\it operators},
{\it separators} and {\it newlines}.

\paragraph{Identifiers}
\label{sec:identifiers}
An identifier consists of a letter followed by other letters, 
digits and underscores. The letters are the ASCII characters \texttt{a}-\texttt{z} and
\texttt{A}-\texttt{Z}. Digits are ASCII characters \texttt{0}-\texttt{9}. SMURF is case sensitive.

\begin{grammar}
<letter> $\rightarrow$ [`a'-`z' `A'-`Z'] 

<digit> $\rightarrow$ [`0'-`9'] 

<underscore> $\rightarrow$ {`_'} 

<identifier> $\rightarrow$ <letter> (<letter> | <digit> | <underscore>)*
\end{grammar}

\paragraph{Keywords}
\label{sec:keywords}
Keywords in SMURF are identifiers reserved by the language. Thus, they are not available for
re-definition or overloading by users. 

\begin{table} [H]
	\centering
    \begin{tabular}{ll}
    \hline\hline
    Keywords & Descriptions \\ 
    \hline\hline
      \texttt{Bool} & Boolean data type \\ \hline
      \texttt{Int} & Integer data type \\ \hline
      \texttt{Note} & Atomic musical data type \\ \hline
      \texttt{Beat} & Note duration data type\\ \hline
      \texttt{Chord} & Data type equivalent to \texttt{[Note]} type \\ \hline
      \texttt{System} & Data type equivalent to \texttt{[Chord]} type \\ \hline
      \texttt{let, in} & Allow local bindings in expressions  \\ \hline
      \texttt{if, then, else} & Specify conditional expression, else compulsory  \\ \hline
      \texttt{main} & Specify the value of a SMURF program\\ \hline
    \end{tabular}
\end{table}


\paragraph{Constants}
\label{sec:constants}
In SMURF, constants are expressions with a fixed value. Integer literals and
Boolean keywords are the constants of SMURF. 

\setlength{\grammarindent}{6em}
\begin{grammar}
<digit> $\rightarrow$ [`0'-`9'] 

<constant> $\rightarrow$ \texttt{-}? [`1'-`9'] <digit>* \\
												 <digit>+ \\
												\texttt{True} \\
												\texttt{False}
\end{grammar}

\paragraph{Operators}
SMURF permits arithmetic, comparison, boolean, list, declaration, and row operations, all of which
are carried out through the use of specific operators. The syntax and semantics of all of these
operators are described in sections \ref{sec:postfixop}, \ref{sec:prefixop}, and \ref{sec:binaryop},
except for declaration operators, which are described in section \ref{sec:declarations}.


\paragraph{Newlines}
SMURF uses newlines to signify the end of a declaration, except
when included in a comment or preceded by the \texttt{\textbackslash} token. In the latter case, the newline is ignored by the compiler 
(see example below). If no such token precedes a newline, then the compiler will treat the newline as
a token being used to terminate a declaration.

\paragraph{Separators}

\begin{grammar}
<separator> $\rightarrow$ \texttt{,} \\
												  \texttt{\&} \\
													\texttt{\textbackslash}
\end{grammar}

Separators in SMURF are special tokens used to separate other tokens. 
Commas are used to separate elements in a list.
The \texttt{\&} symbol can be used in place of a newline. That is, the compiler
will replace all \texttt{\&} characters with newlines. The
\texttt{\textbackslash} token, when followed by a newline token,
may be used to splice two lines. E.g.
\begin{lstlisting}
genAltChords (x:y:ys) = [(x,Time 4,1)]   \
                        :[(y,Time 4,-1)]:(genAltChords ys)
\end{lstlisting}
is the same as 
\begin{lstlisting}
genAltChords (x:y:ys) = [(x,Time 4,1)]:[(y,Time 4,-1)]:(genAltChords ys)
\end{lstlisting}

The \texttt{\&} and \texttt{\textbackslash} tokens are syntactic sugar and exist solely for code formatting when writing a SMURF program.

\subsubsection{Whitespace}
\label{sec:whitespaces}
Whitespace consists of any sequence of {\it blank} and {\it tab} characters.
Whitespace is used to
separate tokens and format programs. All whitespace is ignored by the
SMURF compiler. As a result, indentations are not significant in SMURF.

\subsection{Meaning of Identifiers}
In SMURF, an identifier is either a keyword or a name for a variable or a function. 
The naming rules for identifiers are defined in section~\ref{sec:identifiers}. This section outlines
the use and possible types of non-keyword identifiers.


\subsubsection{Purpose}
\paragraph{Functions}
Functions in SMURF enable users to structure programs in a more modular way. 
Each function takes at least one argument and returns exactly one value (except the built in \texttt{random} function, see section
~\ref{sec:random} for more details), whose types need to be 
explicitly defined by the programmer. The function describes how to produce the return value,
given a certain set of arguments.
SMURF is a side effect free language, which means
that if provided with the same arguments, a function is guaranteed to return the same value (again, this is no longer the case
when using the \texttt{random} function). 


\paragraph{Variables}
In SMURF, a variable is an identifier that is bound to a constant value
or to an expression. Any use of a variable within the scope of its definition
refers to the value or expression to which the variable was bound.
Each variable has a static type which can be automatically
deduced by the SMURF compiler, or explicitly defined by users. The variables in SMURF are immutable.


\subsubsection{Scope and Lifetime}
The lexical scope of a top-level binding in a SMURF program is the whole program itself.
As a result of this fact, a top-level binding can refer to any other top-level variable or function
on its right-hand side, regardless of which bindings occur first in the program.
Local bindings may also occur
with the \texttt{let} \emph{declarations} \texttt{in} \emph{expression} construct, and the
scope of a binding in \emph{declarations} is \emph{expression} and the right hand side of any other bindings in
\emph{declarations}. A variable or function is only visible within its scope.
An identifier becomes invalid after the ending of its scope. E.g.

\begin{verbatim}
prime = [2,0,4,6,8,10,1,3,5,7,9,11] 
main = let prime = [0,2,4,6,8,10,1,3,5,7,9,11]
           p3 = (head prime) + 3
       in (p3, 0)$4
\end{verbatim}

In line 1, \texttt{prime} is bound to a list of integers in a top-level definition, so it
has global scope.
In line 2, the \texttt{main} identifier (a special keyword described in \ref{sec:main}) 
is bound to a \texttt{let} expression. The \texttt{let} expression declares two local variables,
\texttt{prime} and \texttt{p3}. In line 3, the \texttt{head} function looks for a definition
of prime in the closest scope, and thus uses the binding in line 2. So the result
of the expression in line 4 is \texttt{(3,0)\$4}. After line 4 and prior to line 2, 
the locally defined \texttt{prime} and
\texttt{p3} variables are invalid and can't be accessed.

\subsubsection{Basic Types}
\label{sec:basictype}
There are three fundamental types in SMURF: \texttt{Int}, \texttt{Bool} and \texttt{Beat}. 
\begin{itemize}
\item \texttt{Int}: integer type
\item \texttt{Bool}: boolean type
\item \texttt{Beat}: beat type, used to represent the duration of a note. A constant of type
			\texttt{Beat} is any power of 2 ranging from 1 to 16. These beat constants are assumed to be
			of type \texttt{Int} until they are used in an operation that requires them to have type \texttt{Beat} e.g.
			when used as an operand to the beat arithmetic operator \texttt{\$+}.
\end{itemize}

\subsubsection{Structured Types}
\label{sec:structtype}
Structured types use special syntactic constructs and other types to describe new types. 
There are two structured types in SMURF: {\it list} types and {\it function} types.

A {\it list} type has the format \texttt{[}\emph{t}\texttt{]} where \emph{t} is a type that specifies the type
of all elements of the list. Thus, all elements of a list of type \texttt{[}\emph{t}\texttt{]}
must themselves have type \emph{t}. Note that \emph{t} itself may be a list type.

A {\it function} type has the format \emph{t$_1$} \texttt{->} \emph{t$_2$} \texttt {->}
\dots \texttt{->} \emph {t$_n$} \texttt{->} \emph{t$_{ret}$} which specifies a function type
that takes $n$ arguments, where the $kth$ argument has type \emph{t$_k$}, and returns an expression
of type \emph{t$_{ret}$}. Any type may be used to define a function type, except for a function
type itself. In other words, functions may not be passed as arguments to other functions, nor may
a function return another function.


\subsubsection{Derived Types}
\label{sec:derivedtype}
Besides the basic types, SMURF also has several derived types. 

Expressions of type \texttt{Note} are used to represent musical notes in SMURF.
The note type can be written as

\vspace{1em}
(\emph{Int}, \emph{Int})\texttt{\$}\emph{Beat}[\texttt{.}]* 
\vspace{1em}

The first expression of type \texttt{Int} must evaluate to an integer in the range from -1 to 11,
representing a pitch class or a rest. 
When this expression evaluates to -1, the note is treated as a rest, otherwise it represents the pitch class of 
the note. The second expression of type \texttt{Int} must evaluate to an integer in the range of 0-3,
representing the register of the note, where the integer values and corresponding registers are given below.
\begin{itemize}
    \item 1: Bass clef, B directly below middle C to first C below middle C
    \item 0: Bass clef, next lowest B to next lowest C
    \item 2: Treble clef, middle C to the first B above middle C
    \item 3: Treble clef, first C above middle C to next highest B
\end{itemize}
The expression of type \texttt{Beat} refers to the duration of the note, and may be followed by
optional dots. The dot is a postfix operator described in section \ref{sec:postfixop}. Ignoring the possible
postfix operators, the expression must evaluate to an integer in the range [1,2,4,8,16].
Using this format, a quarter note on middle C could be written as \texttt{(0,2)\$4}.

The \texttt{Chord} type is used to represent several notes to be played simultaneously.
It is equivalent to the list type \texttt{[Note]}.
The compiler will check to make sure all the notes in a chord have the same time duration.

The \texttt{System} type is used to represent a list of chords to be played sequentially.
It is equivalent to the list type \texttt{[Chord]}.

\subsubsection{Polymorphic Types}
\label{sec:polymorphism}

SMURF provides the option of specifying an identifier as having a polymorphic type by using a non-keyword identifier in place of a
basic, structured, or derived type in that identifier's type signature. For more information on the structure of type signatures,
see section ~\ref{sec:typesig}.
For example, \texttt{a :: b} specifies that a variable named \texttt{a} has polymorphic type \texttt{b}, where
\texttt{b} can be replaced with any basic, structured, or derived type. Using the same polymorphic type across different type signatures
is permitted and each use has no bearing on another. For example, giving \texttt{a :: b} and \texttt{c :: b} merely states that \texttt{a}
and \texttt{c} are both variables with polymorphic types and would be equivalent to giving \texttt{a :: hippo} and \texttt{c :: dinosaur}.
However, if the same identifier is used multiple times as a type in a function's type signature, then the types assigned to those components
of the function must be identical. For example, say we have a function \\\\ \texttt{f :: Int -> b -> [b]} \\\\ This type signature specifies that
\texttt{f} takes two arguments, the first of type \texttt{Int} and the second of polymorphic type, and that the expression bound to \texttt{f}
must be a list type, where the elements of the list are of the same type as the second argument passed to \texttt{f}. Thus
\texttt{f 0 True = [False]} would be a valid function declaration (as \texttt{True} and \texttt{False} both have type \texttt{Bool}) 
given this type signature, but \texttt{f 0 True = [1]} would result
in a compile-time error because \texttt{1} has type \texttt{Int}.


\subsection{Expressions}

This section describes the syntax and semantics of \emph{expressions} in 
SMURF. Some expressions in SMURF use prefix, infix, or postfix operators.
Unless otherwise stated, all infix and postfix operators are left-associative and
all prefix operators are right-associative. Some examples of association are given below.


%TODO: check the correct position for this table
\begin{center}
	\begin{tabular}{|c|c|}
		\hline
		Expression & Association \\
		\hline
		\texttt{f x + g y - h z} & \texttt{((f x) + (g y)) - (h z)} \\
		\texttt{ let \{ ... \} in x + y} & \texttt{let \{ ... \} in (x + y)} \\
		\texttt{$\sim$ <> [0,1,2,3,4,5,6,7,8,9,10,11]} &
					\texttt{($\sim$ (<> [0,1,2,3,4,5,6,7,8,9,10,11]))}\\
		\hline
	\end{tabular}
\end{center}

%
%\subsubsection{Errors}
%
%Errors can only occur during compilation of a SMURF program. All generated 
%code will run successfully. 
%



\subsubsection{Variable Expression}

\begin{grammar}
<variable-expr> $\rightarrow$ <variable> 

<variable> $\rightarrow$ <identifier> 
\end{grammar}
A variable $x$ is an expression whose type is the same as the type of $x$.
When we evaluate a variable, we are actually evaluating
the expression bound to the variable in the closest lexical scope. A variable is represented with
an \emph{identifier} as defined in section ~\ref{sec:identifiers}.

\subsubsection{Constant Expression}
\begin{grammar}
<constant-expr> $\rightarrow$ <constant> 
\end{grammar}
An integer or boolean constant, as described in section \ref{sec:constants}, is an expression
with type equivalent to the type of the constant.

\subsubsection{Parenthesized Expression}
\begin{grammar}
<parenthesized-expr> $\rightarrow$ \texttt{(} <expression> \texttt{)}
\end{grammar}
An expression surrounded by parentheses is itself an expression. Parentheses can be used to force the evaluation
of an expression before another e.g. \texttt{2 + 3 - 4 - 5} evaluates to \texttt{((2+3) - 4) - 5 = -4}\\ but
\texttt{2 + 3 - (4 - 5)} evaluates to \texttt{(2 + 3) - (4 - 5) = 6}.

\subsubsection{List Expression}

\begin{grammar}
<list-expr> $\rightarrow$ \texttt{[} \texttt{]} \\
													\texttt{[} <expression> (\texttt{,} <expression>)* \texttt{]}
\end{grammar}
A list is an expression. Lists can be written as: \\

\texttt{[}\emph{expression$_{1}$}\texttt{,} \emph{...}\texttt{,} \emph{expression$_{k}$}\texttt{]} \\

	or \\

  \emph{expression$_{1}$}\texttt{:}\emph{expression$_{2}$}\texttt{:} \emph{...} \texttt{:}\emph{expression$_{k}$}\texttt{:[]} \\ \\
where \textit{k} $>=$ 0. These two lists are equivalent.
The expressions in a list must all be of the same type. The empty list \texttt{[]} has a polymorphic type i.e. it can take on the
type of any other list type depending on the context.


\subsubsection{Notes}

\begin{grammar}
<note-expr> $\rightarrow$ \texttt{(}<expression>\texttt{,} <expression>\texttt{)\$}<expression>
\end{grammar}

A note is an expression, and is written as a tuple of expressions of type \texttt{Int} followed
by a \texttt{\$} symbol and an expression of type \texttt{Beat}. The values of each of these
expressions must follow the rules outlined in section \ref{sec:derivedtype}.

\subsubsection{Postfix Operator Expressions}
\label{sec:postfixop}

\begin{grammar}
<postfix-expression> $\rightarrow$ <expression>\texttt{.}
\end{grammar}

The only expression in SMURF using a postfix operator is the partial augmentation of an expression
of type \texttt{Beat}, which uses the dot operator. This operator has higher precedence than any prefix or infix operator.
We say ``partial augmentation" because a dot increases the durational value of
the expression to which it is applied, but only by half of the durational value of that expression.
That is, if \emph{expr} is an expression of type \texttt{Beat} that evaluates to a duration of $n$, 
then \emph{expr}\texttt{.} is a postfix expression of type \texttt{Beat} 
that evaluates to a duration of $n + n/2$. In general, a note with duration $d$ and total dots $n$ has a total
duration of $2d - d/2^n$. The dot operator may be applied until it represents an addition of a sixteenth note
duration, after which no more dots may be applied. For instance, \texttt{4..} is legal, as this
is equivalent to a quarter note duration plus an eighth note duration (the first dot) plus a
sixteenth note duration (the second dot). However, \texttt{8..} is not legal, as the second
dot implies that a thirty-second note duration should be added to the total duration of this
expression. Our compiler checks the number of dots and returns an error if too many are applied.

\subsubsection{Prefix Operator Expressions}
\label{sec:prefixop}
\begin{grammar}
<prefix-expression> $\rightarrow$ <prefix-op> <expression>
\end{grammar}


\begin{table} [H]
\centering
\begin{tabularx}{0.9\textwidth}{llX}
\hline\hline
Prefix Operator & Description & Example \\
\hline\hline
   \texttt{$\sim$} & Tone row inversion & $\sim$ \texttt{row} (returns the inversion of \texttt{row})\\ \hline
   \texttt{<>} & Tone row retrograde & \texttt{<>}~\texttt{row} (returns the
           retrograde of \texttt{row})\\ \hline
   \texttt{!} & Logical negation & \texttt{if !(a == 5) then True else False} \\ \hline
\end{tabularx}
\end{table}

SMURF has three prefix operators: logical negation, tone row inversion, and tone row retrograde.
There is another row transformation operator, but it takes multiple arguments and is described in
section \ref{sec:bintonerow}. The tone row prefix operators have higher precedence than any infix operator, while the logical negation operator
is lower in precedence than all infix operators except for the other logical operators \texttt{\&\&} and \texttt{$||$}.
The logical negation operator can only be applied to expressions of
type \texttt{Bool}, and the two row operators can only be applied to expressions of type
\texttt{[Int]}. The compiler will check that all of the integers in a list are in the range $0-11$
if the list is passed to either of the tone row operators.
All three operators return an expression of the same type as the expression the
operator was applied to.

\subsubsection{Binary Operator Expressions}
\label{sec:binaryop}
\begin{grammar}
<binary-expression> $\rightarrow$ <expression>$_1$ <binary-op> <expression>$_2$
\end{grammar}
The following categories of binary operators exist in SMURF, and are listed in order of decreasing
precedence: list, arithmetic, comparison, boolean, tone row. 

\paragraph{List operators}

\begin{table} [H]
\centering
\begin{tabular}{lll}
\hline\hline
List Operator & Description & Example \\
\hline\hline
   \texttt{++} & List Concatenation & \texttt{[1,2,3] ++ [4,5,6]} (result is \texttt{[1,2,3,4,5,6]}) \\ \hline
   \texttt{:} & List Construction & \texttt{1 : [2,3,4]} (result is \texttt{[1,2,3,4]}) \\ \hline
\end{tabular}
\end{table}

List operators are used to construct and concatenate lists. 
These two operators are \texttt{:} and \texttt{++}, respectively. The \texttt{:} operator has 
higher precedence than the \texttt{++} operator. Both of these operators are right-associative.
The list construction operator requires that  \emph{expression$_{2}$} be an expression of type 
\texttt{[}\emph{t}\texttt{]}, where \emph{t} is the type of \emph{expression$_{1}$}. 
In other words, \emph{expression$_{1}$} must have the same type 
as the other elements in \emph{expression$_{2}$} when doing list construction.
When doing list concatenation, both
\emph{expression$_{1}$} and \emph{expression$_{2}$} must have type \texttt{[}\emph{t}\texttt{]},
where \emph{t} is some non-function type. 

\paragraph{Arithmetic operators}

\begin{table} [H]
\centering
\begin{tabularx}{\textwidth}{lXX}
\hline\hline
Arithmetic Operator & Description & Example \\
\hline\hline
  \texttt{+} & Integer Addition  & \texttt{a + 2} \\ \hline
  \texttt{-} & Integer Subtraction  & \texttt{5 - a} \\ \hline 
  \texttt{*} & Integer Multiplication  & \texttt{5 * 10} \\ \hline 
  \texttt{/} & Integer Division  & \texttt{4 / 2} \\ \hline 
  \texttt{\%} & Integer Modulus, ignores negatives  & \texttt{14 \% 12} \\ \hline
  \texttt{\%+} & Pitch Class Addition (addition mod 12)  & \texttt{14 \%+ 2 == 4 }\\ \hline
  \texttt{\%-} & Pitch Class Subtraction (subtraction mod 12)  & \texttt{14 \%- 2 == 0 } \\ \hline
  \texttt{$\$$+} & Rhythmic Addition & \texttt{2 $\$$+ 2 == 1} \\ \hline
  \texttt{$\$$-} & Rhythmic Subtraction & \texttt{1 $\$$- 2 == 2}  \\ \hline
  \texttt{$\$$*} & Rhythmic Augmentation & \texttt{8 $\$$* 4 == 2}  \\ \hline
  \texttt{$\$$/} & Rhythmic Diminution & \texttt{2 $\$$/ 8 == 16}  \\ \hline
\end{tabularx}
\end{table}

There are three types of arithmetic operators:
basic, pitch class, and rhythmic. 
Basic arithmetic operators are those found in most programming languages like 
\texttt{+}, \texttt{-}, \texttt{*}, \texttt{/}, and \texttt{\%}, which operate on expressions of
type \texttt{Int}. It should be noted that the modulus operator ignores negatives e.g.
\texttt{13 \% 12} is equal to \texttt{-13 \% 12} is equal to \texttt{1}.
The pitch class operators are \texttt{\%+} and \texttt{\%-}. These can be read as mod 12 addition
and mod 12 subtraction. They operate on expressions of type \texttt{Int}, but the expressions
must evaluate to values in the range $0-11$. The built-in mod 12
arithmetic serves for easy manipulation of pitch class integers.
Lastly, there are rhythmic arithmetic operators (both operands must be of type \texttt{Beat}). 
These include \texttt{\$+}, \texttt{\$-}, \texttt{\$*}, and \texttt{\$/}. If one of the operands
of these operators is of type \texttt{Int}, it will be cast to a \texttt{Beat} type if it is an
allowable power of 2 and generate a semantic error otherwise.

In terms of precedence, \texttt{*}, \texttt{/}, \texttt{\$*}, \texttt{\$/} and \texttt{\%} are all at the same
level of precedence, which is higher than the level of precedence shared by the rest of the arithmetic operators.

\paragraph{Comparison operators}

\begin{table} [H]
\centering
\begin{tabularx}{\textwidth}{XlX}
\hline\hline
Comparison Operator & Description & Example \\
\hline\hline
  \texttt{<}  & Integer Less than & \texttt{if a < 5 then True else False} \\ \hline
  \texttt{>}  & Integer Greater than & \texttt{if a > 5 then True else False}  \\ \hline
  \texttt{<=}  & Integer Less than or equal to & \texttt{if a <= 5 then True else False} \\ \hline
  \texttt{>=} & Integer Greater than or equal to & \texttt{if a >= 5 then True else False} \\ \hline
  \texttt{\$<} & Rhythmic Less than & \texttt{4 \$< 8 == False} \\ \hline
  \texttt{\$>}  & Rhythmic Greater than &  \texttt{4 \$> 8 == True}  \\ \hline
  \texttt{\$<=} & Rhythmic Less than or equal to & \texttt{4 \$<= 4 == True} \\ \hline
  \texttt{\$>=} & Rhythmic Greater than or equal to &  \texttt{1 \$>= 16 == True} \\ \hline
  \texttt{==} & Structural comparison & \texttt{if a == 5 then a = True else a = False} \\ \hline
\end{tabularx}
\end{table}

SMURF allows comparison operations between expressions of type \texttt{Int} or \texttt{Beat}.
Structural comparison, however, can be used to compare expressions of any type for equality.
All of the comparison operators have the same precedence except for structural comparison, which
has lower precedence than all of the other comparison operators.

\paragraph{Boolean operators}


\begin{table} [H]
\centering
\begin{tabular}{lll}
\hline\hline
Boolean Operator & Description & Example \\
\hline\hline
   \texttt{\&\&} & Logical conjunction & \texttt{if b \&\& c  then True else False} \\ \hline
   \texttt{\textbar\textbar} & Logical disjunction & \texttt{if b \textbar\textbar\space   c  then True else False} \\ \hline
 \end{tabular}
\end{table}

Boolean operators are used to do boolean logic on expressions of type \texttt{Bool}. Logical
conjunction has higher precedence than logical disjunction.

\paragraph{Tone row operators}
\label{sec:bintonerow}
The only binary tone row operator is the transposition operator, \texttt{$\wedge\wedge$}.
\emph{expression$_1$} must have type \texttt{Int}, and \emph{expression$_2$} must be an 
expression that evaluates to a list of pitch classes. The result of this operation is a new
tone row where each pitch class has been transposed up by $n$ semitones, where $n$ is the result
of evaluating \emph{expression$_2$}.

\subsubsection{Conditional expressions}
\begin{grammar}
<conditional-expression> $\rightarrow$ \texttt{if} \emph{expression$_{boolean}$} \texttt{then}
													\emph{expression$_{true}$} \texttt{else} 
													\emph{expression$_{false}$}
\end{grammar}
When the value of \emph{expression$_{boolean}$} evaluates to true, \emph{expression$_{true}$} 
is evaluated, otherwise \emph{expression$_{false}$} is evaluated. \emph{expression$_{boolean}$}
must have type \texttt{Bool}.

\subsubsection{Let Expressions}
\begin{grammar}
<let-exp> $\rightarrow$ \texttt{let} <decls>+ \texttt{in} <expression>
\end{grammar}
Let expressions have the form \texttt{let} \emph{decls} \texttt{in} \emph{e}, where 
\emph{decls} is a list of one or more declarations and \emph{e} is an expression. 
The scope of these declarations is discussed in section \ref{sec:declarations}.

The declarations in a let expression must be separated by either the \texttt{\&} symbol or by the newline character. For example:

\begin{verbatim}
let x = 2 & y = 4 & z = 8 
in x + y + z
\end{verbatim}

The previous code is equivalent to the following:

\begin{verbatim}
let x = 2
    y = 4
    z = 8
in x + y + z
\end{verbatim}

If the first code snippet were written without the \& symbol and no newlines after each declaration, a compile-time error will be raised.

\subsubsection{Function application expressions}

\begin{grammar}
<function-app-expression> $\rightarrow$ <identifier> <expression>+
\end{grammar}

A function gets called by invoking its name and supplying any necessary arguments. 
Functions can only be called if they have been declared in the same scope where the call occurs,
or in a higher scope. Functions may be called recursively. Function application associates 
from left to right. Parentheses can be used to change the precedence from the default. Furthermore,
parentheses must be used when passing the result of a complex expression to a function. Here, complex expression
refers to any expression that uses an operator or itself is a function call.
The following evaluates function \emph{funct1} with argument \emph{b} then evaluates function
\emph{funct2} with argument \emph{a} and the result from evaluating (\emph{funct1 b}): \\
    
    \emph{funct2 a (funct1 b)}\\ \\
If the parentheses were not included, a compile-time error would be generated, as it would imply
that \emph{funct2} would be called with \emph{a} as
its first argument and \emph{funct1} as its second argument, which is illegal based on the description
of function types in section \ref{sec:structtype}.

A function call may be used in the right-hand side of a binding just like any other expression. 
For example:
\begin{verbatim}
let a = double 10 
in a
\end{verbatim}
evaluates to 20, where \texttt{double} is a function that takes a single integer argument and
returns that integer multiplied by two. 

%\subsubsection{Guards}
%\begin{grammar}
%<guard> $\rightarrow$  let <decls> | <infixexp>          
%\end{grammar}
%
%There are two kinds of guards in SMURF: local bindings and boolean guards. Local bindings have the form let $<decls>$ and introduces the declarations to the program environment. Boolean guards are expressions of type Bool. The boolean guard succeeds if the expression evaluates to True.
%
\subsection{Declarations and Bindings}
\label{sec:declarations}
\setlength{\grammarindent}{4em}

This section of the LRM describes the syntax and informal semantics of
declarations in SMURF. A program in SMURF, at its top-most level, is a series of declarations
separated by newline tokens.
Declarations may also occur inside of
\texttt{let} expressions (but still must be separated with newline tokens).
The scoping of such declarations is described 
in this section. There are three types of declarations in SMURF: 
type signatures, definitions, and function declarations.

\begin{table} [H]
\centering
\begin{tabularx}{\textwidth}{lXl}
\hline\hline
Declaration Operator & Description & Example \\
\hline\hline
   \texttt{::} & Type specification & \texttt{number :: Int} \\ \hline
   \texttt{->} & Argument and function return type specification
     & \texttt{isPositiveNum :: Int -> Bool}  \\ \hline
	 \texttt{=} & Variable or function binding & \texttt {x = 3} \\ \hline
\end{tabularx}
\end{table}

\subsubsection{Type Signatures}
\label{sec:typesig}

\begin{grammar}

<type-sig> $\rightarrow$ <identifier> \texttt{::} (<type>|<function-type>)

<function-type> $\rightarrow$ <type> \texttt{->} <type> (\texttt{->} <type>)*

<type> $\rightarrow$ \texttt{Int} \\ \texttt{Bool} \\ \texttt{Beat} \\ \texttt{Note} \\
											\texttt{Chord} \\ \texttt{System} \\ <identifier> \\ 
											\texttt{[} <type> \texttt{]} \\ 
										

\end{grammar}

A type signature explicitly defines the type for a given identifier. The
\texttt{::} operator can be read as ``has type of." Only one type signature
for a given identifier can exist in a given scope. That is, two different
type signatures for a given identifier can exist, but they must be declared
in different scopes. There are four categories of types in SMURF: basic,  structured, derived, 
and polymorphic types; types are described in sections~\ref{sec:basictype}-\ref{sec:polymorphism}.

\subsubsection{Definitions}

\begin{grammar}

<definition> $\rightarrow$ <identifier> $=$ <expression>

\end{grammar}

A definition binds an identifier to an expression. All definitions at a given
scope must be unique and can be mutually recursive. For example, the following
is legal in SMURF:

\begin{verbatim}
let x = 4
  	z = if y == 7 then x else y
  	y = let x = 5 
        in x + 3 
in x + z + y
\end{verbatim}


The $x$ in the nested let expression is in a different scope than the $x$ in the global let expression, 
so the two definitions do not conflict. $z$ is able to refer to $y$ even though
$y$ is defined after $z$ in the program. In this example, the three 
identifiers $x, y, $ and $z$ in the global \texttt{let} will evaluate to values 4, 8, and 8, respectively,
while the identifier $x$ in the nested let expression will evaluate to 5.

A type signature may be given for the identifier in a definition but is not required.

\subsubsection{Function Declarations}

\setlength{\grammarindent}{5em}
\begin{grammar}

<fun-dec> $\rightarrow$ <identifier> <args> \texttt{=} <expression>

<args> $\rightarrow$ <pattern> \\ <pattern> <args> 


<pattern> $\rightarrow$ <pat> \\ <pat> \texttt{:} <pattern> \\ \texttt{[} <pattern-list>$?$ \texttt{]} \\
												\texttt{(} <pattern> \texttt{)} 

<pattern-list> $\rightarrow$ <pat> (\texttt{,} <pat>)* 

<pat> $\rightarrow$ <identifier> \\ <constant> \\ \texttt{_}
												
\end{grammar}

A function declaration defines an identifier as a function that takes
some number of expressions as arguments and, based on which patterns are matched
against those expressions when the function is called, returns the result
of a given expression. Essentially, a function declaration can be seen as a 
binding associating an expression with a function identifier and a set of patterns that will be matched
against the function's arguments when the function is called.
There must be at least one pattern listed
as an argument in a function declaration. All function declarations for the same identifier
in a given scope must have the same number of patterns given in the declaration.

Unlike variable definitions, multiple function declarations for the same identifier may exist
in the same scope, as long as no two declarations have an equivalent set of patterns. This rule does not pertain
to multiple function declarations for an identifier across different scopes.

If a function declaration for some identifier $x$ occurs in scope $n$, then
a type signature for $x$ in scope $k>=n$ is required. That is if a function has
been declared but its type has not been explicitly stated in the same or a higher
scope, a compile-time error will be generated. The type of the patterns in a function declaration
are checked at compile-time as well, and an error
is issued if they don't match the types specified in that function's 
type signature.

A \emph{pattern} can be used in a function declaration to ``match" against arguments passed to the function. The
arguments are evaluated and the resultant values are matched against the patterns in the same order they were given
to the function. If the pattern
is a constant, the argument must be the same constant or evaluate to that constant value in order for a match to
occur. If the pattern is an identifier, the argument's value is bound to that identifier in the scope of the
function declaration where the pattern was used. If the pattern is the wildcard character `\texttt{_}', 
any argument will be matched and no binding will occur. If the pattern is structured, the argument must follow
the same structure in order for a match to occur. 

Below, we have defined an example function \texttt{f} that takes two arguments. 
The value of the function call is dependent
on which patterns are matched. The most restrictive patterns are checked against the arguments first.
In this example, we first check if the first argument evaluates to \texttt{0} (we disregard the second
argument using the wildcard character), and return \texttt{True} if it does. Otherwise, we check if the second argument
evaluates to the empty list, and, if so, return \texttt{False}. Next, we check if the second argument evaluates to a list
containing exactly two elements and, if so, the first element is bound to \texttt{x} and the second is bound to
\texttt{y} in the expression to the right of the binding operator \texttt{=}, and that expression is evaluated and
returned. Finally, if
none of the previous pattern sets matched, we bind the first argument to \texttt{m}, the head of the
second argument to \texttt{x}, and the rest of the second argument to \texttt{rest}. Note we can do this
as we already checked if the second argument was the empty list, and, since we did not match that pattern,
we can assume there is at least one element in the list.

\begin{verbatim}
f :: Int -> [Int] -> Bool
f _ [] = False
f _ [x, y] = if x then True else False
f 0 _ = True
f m x:rest = f m rest 
\end{verbatim}


\subsubsection{\texttt{main} Declaration}
\label{sec:main}
Every SMURF program must provide a definition for the reserved identifier \texttt{main}. This
identifier may only be used on the left-hand side of a top-level definition. The expression
bound to \texttt{main} is evaluated and its value is the value of the SMURF program itself.
That is, when a SMURF program is compiled and run, the expression bound to \texttt{main} is
evaluated and the result is converted to our bytecode representation of a MIDI file. As a result,
this expression must evaluate to a value of type \texttt{[]}, \texttt{Note}, \texttt{Chord}, 
\texttt{System}, or \texttt{[System]} (or any type that is equivalent to one of these). If a definition for \texttt{main} is not included in a 
SMURF program or if the expression bound to it does not have one of the types outlined above,
a compile-time error will occur.

\subsection{Library Functions}

Below are the library functions that can be used in the SMURF language. 
While some of these functions are implemented using SMURF, others (such as print) are special cases that provide
helpful services but cannot explicitly be defined in our language. These special cases are implemented in the translation
section of the compielr. Each library function will include a description 
and its SMURF definition (if it can be defined using SMURF). Users are not permitted to redefine any of these functions. \newline

\noindent\textbf{Print}

The function \texttt{print} takes an argument of any type, evaluates it, and prints the result to standard output.
The result of calling the \texttt{print}
function is the result of evaluating its argument i.e. \texttt{print(x+1)} evaluates to \texttt{x+1}.\\

\noindent\textbf{Random}
\label{sec:random}

The function \texttt{random} is the only SMURF function that takes no parameters. It is another example of a function that cannot
be explicitly defined using the SMURF language. The result of a call to \texttt{random} is a pseudo-random integer between 1 and 1000000, inclusive.
For example, \texttt{random $\%$ 12 } will return some number between 0 and 11.
Every time \texttt{random} is called, a new pseudo-random seed is used to initialize the random number generator used by the compiler, allowing for
different results on each run of a program where random is used. There is no capability for the user to set their own initializing seed. \\

\noindent\textbf{Head}

The function \texttt{head} takes a list as an argument and returns the first element. 
This function is commonly used when working with lists.

\begin{verbatim}
head :: [a] -> a
head (h:tl) = h
\end{verbatim} 


\noindent\textbf{Tail}

The function \texttt{tail} takes a list as an argument and returns the same list with
the first element removed.
This function is commonly used when working with lists. 

\begin{verbatim}
tail :: [a] -> [a]
tail (h:tl) = tl
\end{verbatim} 


\noindent\textbf{MakeNotes}

The function \texttt{makeNotes} takes in three lists and returns a list 
of notes. The first list consists of expressions of type \texttt{Int} representing 
pitches and/or rests. The second list consists of expressions of type \texttt{Int}
representing the register that the pitch will be played in. The third list is a list 
of expressions of type \texttt{Beat} representing a set of durations.  This function allows the user 
to manipulate tone rows independently of beats and registers, then combine the three components
into a list of notes. If the lengths of the three arguments are not equivalent, this function
will only return the list of notes generated from the first $n$ elements of each list, where $n$ is the
length of the shortest list given as an argument.

\begin{verbatim}
makeNotes :: [Int] -> [Int] -> [Beat] -> [Note]
makeNotes [] _ _ = []
makeNotes _ [] _ = []
makeNotes _ _ [] = []
makeNotes (h1:tl1) (h2:tl2) (h3:tl3) = (h1,h2)$h3:(makeNotes tl1 tl2 tl3)
\end{verbatim}


\noindent \textbf{Reverse}

The function \texttt{reverse} takes a list as an argument and returns the same list in reverse.

\begin{verbatim}
reverse :: [a] -> [a]
reverse [] = []
reverse a:rest = (reverse rest) ++ [a]
\end{verbatim}


\noindent \textbf{Last}

The function \texttt{last} takes a list as an argument and returns the last element in the list.

\begin{verbatim}
last :: [a] -> a
last a:[] = a
last a:rest = last rest
\end{verbatim}

\noindent \textbf{Drop}

The function \texttt{drop} takes an integer $n$ and a list as arguments, and returns the same list
with the first $n$ elements removed.

\begin{verbatim}
drop :: Int -> [a] -> [a]
drop 0 x = x
drop _ [] = []
drop x l:rest = drop (x - 1) rest
\end{verbatim}

\noindent \textbf{Take}

The function \texttt{take} takes an integer $n$ and a list as arguments, and returns a list composed
of the first $n$ elements of the original list.

\begin{verbatim}
take :: Int -> [a] -> [a]
take 0 _ = []
take _ [] = []
take x l:rest = l : (take (x - 1) rest)
\end{verbatim}
