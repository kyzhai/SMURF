\documentclass[dvips, 12pt]{article}

% Any percent sign marks a comment to the end of the line

% Every latex document starts with a documentclass declaration like this
% The option dvips allows for graphics, 12pt is the font size, and article
%   is the style


\usepackage[margin=1in]{geometry}

\usepackage{float}
\restylefloat{table}

\usepackage{tabularx}
\usepackage{graphicx}
\usepackage{url}
\usepackage{harmony}
\usepackage{musixtex}
\usepackage{subfigure}
\usepackage{caption}
\usepackage{listings}
\lstset{language=Haskell}
\usepackage{syntax}
\usepackage{comment}


\usepackage{color}
%\usepackage{biblatex}
%\usepackage{subcaption}

% These are additional packages for "pdflatex", graphics, and to include
% hyperlinks inside a document.


% These force using more of the margins that is the default style


% Everything after this becomes content
% Replace the text between curly brackets with your own

\title{{\Huge \bfseries SMURF Language Reference Manual} \\ \Large \it Serial MUsic Represented as Functions \vspace{0.6cm}}

\author{\normalsize Richard Townsend, Lianne Lairmore, Lindsay Neubauer, Van Bui, Kuangya Zhai
	\\ \small \{rt2515, lel2143, lan2135, vb2363, kz2219\}@columbia.edu \vspace{0.6cm}}

\date{\today \vspace{2cm}}

% You can leave out "date" and it will be added automatically for today
% You can change the "\today" date to any text you like

\begin{document}
\maketitle
\clearpage


% This command causes the title to be created in the document
\tableofcontents

\section{Syntax Notation}
The syntax notation used in this manual are as follows. Syntactic 
categories are indicated by \emph{italic} type. Literal words and 
characters will be displayed in \texttt{typeset}. Alternatives are listed 
on separate lines. Optional terminal and non-terminals are indicated by the
subscript\_{opt}. 

Regular expression notations are used to specigy grammar patterns in this manual.
r* means the pattern r may appear zero or more time, r+ means the r may appear one or more 
times, r? means r may appear zero or once. r1 | r2 denotes an option between two patterns, r1 r2 denotes 
r1 followed by r2. 


\section{Lexical Conventions}
\subsection{A High Level Description of SMURF Programs}
SMURF is a function language that allows a composer to create serialist music
based on twelve-tone composition. In general, serialism is a musical composition
technique where a set of values, chosen through some methodical progress,
          generates a sequence of musical elements. SMURF is based on the
          functional syntax and semantics set forth by Haskell. The backend of
          SMURF generates MIDIs corresponding to the uses's initial program in
          SMURF. 

\subsection{Tokens}
SMURF has 5 kinds of tokens: identifiers, keywords, constants, operators and whitespaces.

\subsubsection{Identifiers}
\label{sec:identifiers}
An identifier consists of a letter followed by other letters and
digits. The letters are the ascii characters a-z, A-Z and \_. Digits are ascii
characters 0-9. SMURF is case sensitive.
\begin{verbatim}
letter -> [`a'-`z' `A'-`Z']
digit -> [`0'-`9']
identifiers -> letter (letter | digit | `_')*
\end{verbatim}

\subsubsection{Keywords}
This is a list of reserved keywords in SMURF. The keywords are used by the
language, thus are not avaliable for re-definition or overloading.
\begin{table} [H]
	\centering
    \begin{tabularx}{0.9\textwidth}{l@{\hskip 3cm}l}
    \hline\hline
    Keywords & \\ 
    \hline\hline
      let & Specify variables and functions  \\ \hline
      in & Allow local variable binding in expression \\ \hline
      if, then, else & Specify conditional expression, else compulsory  \\ \hline
      True, False & Specify boolean logic \\ \hline
      otherwise & Specify conditional expression used with guards \\ \hline 
      %genScore & Generate musical score given list of measures as argument  \\ \hline
    \end{tabularx}
\end{table}


\subsubsection{Constants}
\label{sec:constants}
In SMURF, constants are expressions with a fixed value. Integer literials and
boolean keywords are the constants of SMURF. 

\begin{verbatim}
letter -> [`a'-`z' `A'-`Z']
digit -> [`0'-`9']
constants -> -? [`1'-`9'] digit* | True | False
\end{verbatim}


\subsubsection{Operators}
Operators in SMURF can be classied into comment operators, arithmetic operators, comparision
operators, boolean operators, list operators and function operators. 

SMURF allows nested, multiline comments in addition to single line comments.
\begin{table} [H]
\centering
\begin{tabularx}{\textwidth}{lXl}
\hline\hline
Comment Operator & Description & Example \\
\hline\hline
  /* */ & Multiline comments, nesting allowed & /* This /* is all */ commented */ \\ \hline
  // & Single-line comment & // This is a comment \\ \hline
\end{tabularx}
\end{table}

SMURF allows assignment and addition, subtraction, and modulus on expressions that evaluate to integers. These operators are all infix. The modulus operator ignores negatives e.g. 
\begin{verbatim} 13 % 12 is equal to -13 % 12 \end{verbatim}
\begin{table} [H]
\centering
\begin{tabularx}{\textwidth}{lXX}
\hline\hline
Arithmetic Operator & Description & Example \\
\hline\hline
  = & Assignment operator & a = 4 \\ \hline
  + & Integer arithmetic: plus  & a + 2 \\ \hline
  $-$ & Integer arithmetic: minus  & 5 $-$ a \\ \hline 
  \% & Integer arithmetic: modulus, ignores negatives  & 14 \% 12 \\ \hline
  $\wedge+$ & Beat add operator & 2 $\wedge+$ 2 = 1 (The add of two half notes evaluates one whole note) \\ \hline
  $\wedge-$ & Beat substract operator & 1 $\wedge-$ 2 = 2 (One whole note substracts one half note evaluates one half note) \\ \hline
\end{tabularx}
\end{table}

SMURF allows comparison operations between expressions that evaluate to integers.
\begin{table} [H]
\centering
\begin{tabular}{lll}
\hline\hline
Comparison Operator & Description & Example \\
\hline\hline
  \textless  & Less than & if a \textless\space  5 then True else False \\ \hline
  \textgreater  & Greater than & if a \textgreater\space  5 then True else False  \\ \hline
  \textless=  & Less than or equal to & if a \textless= 5 then True else False \\ \hline
  \textgreater= & Greater than or equal to & if a \textgreater= 5 then True else False \\ \hline
\end{tabular}
\end{table}


SMURF allows logical negation, conjunction, and disjunction, in addition to structural comparison and boolean notation for use with guards.
\begin{table} [H]
\centering
\begin{tabular}{lll}
\hline\hline
Boolean Operator & Description & Example \\
\hline\hline
   == & Structural comparison & if a == 5 then a = True else a = False \\ \hline
   \textcolor{red}{not} & Logical negation & if not a == 5 then True else False \\ \hline
   \&\& & Logical conjunction & if b \&\& c  then True else False \\ \hline
   \textbar\textbar & Logical disjunction & if b \textbar\textbar\space   c  then True else False \\ \hline
 \end{tabular}
\end{table}

SMURF allows concatenation and construction of lists.
\begin{table} [H]
\centering
\begin{tabular}{lll}
\hline\hline
List Operator & Description & Example \\
\hline\hline
   ++ & Concatenation: concat & [1,2,3] ++ [4,5,6] (result is [1,2,3,4,5,6]) \\ \hline
   : & Construction: cons & 1 : [2,3,4] (result is [1,2,3,4]) \\ \hline
   $[]$ & List constructor & $[]$ (result is an empty list) \\ \hline
\end{tabular}
\end{table}


SMURF allows type, argument, and function return type specificatio in addition
to concatenation and construction operations.
\begin{table} [H]
\centering
\begin{tabularx}{\textwidth}{lXl}
\hline\hline
Function Operator & Description & Example \\
\hline\hline
   :: & Type specification & returnIntFunc :: Int \\ \hline
   \textendash\textgreater & Argument and function return type specification
     & isPositiveNum :: Int \textendash\textgreater\space Bool  \\ \hline
   \textbar & Boolean operator used with guards & isSeven num :: [Int] \textendash\textgreater\space Bool\\ 
     && \textbar\space (num == 7) = True \\
     && \textbar\space otherwise = False\\ \hline
\end{tabularx}
\end{table}

SMURF allows 3 transformation operations to tone rows: inversion, retrograde and
transposition.
\begin{table} [H]
\centering
\begin{tabularx}{0.9\textwidth}{llX}
\hline\hline
Tone Row Operator & Description & Example \\
\hline\hline
   $\sim$ & Inversion & $\sim$ {\it row} (returns the inversion of {\it row})\\ \hline
   \textless\textgreater & Retrograde & \textless\textgreater~{\it row} (returns the
           retrograde of {\it row})\\ \hline
   $\wedge\wedge$ & Transposition & $\wedge\wedge$ {\it row} 3 (transposes {\it row} by 3
           semitones)\\ \hline
\end{tabularx}
\end{table}


\subsubsection{Whitespace}
Blanks, tabs and newlines are referred to as whitespaces in SMURF. SMURF treats
\texttt{newlines} as tokens and ignores others whitespaces. 


\subsection{Separators}
SMURF uses separators to seperate tokens. The separators of SMURF include:
\begin{verbatim} 
    ,   :   ;   {   }   whitespaces
\end{verbatim}


\section{Meaning of Identifiers}
In SMURF, identifiers refer to functions and variables.

\subsection{What Are They Need For}
\subsubsection{Functions}
Functions in SMURF enable us to structure our programs in a more modular way. 
A function is a group of statements that is executed when it is called from some
point of the program. 

\subsubsection{variables}
In SMURF, a variable is an abstraction of a computer memory cell or a collection
of memory cells. 
SMURF is a strongly typed programming language, which means the type of a variable can't
be changed once declared. Each variable has a static type which is automatically
deduced by the SMURF compiler. The variables in SMURF are immutable.

\subsection{Scope and Lifetime}
In SMURF, a variable is bound in its scope to a value using constructs like
\texttt{let} or list comprehensions. A variable is visible within its scope.
There is no global variables in SMURF. A variable becomes invalid after the 
ending of its scope.

\subsection{Basic Types}
There are two fundamental types in SMURF: int and bool. 
\begin{itemize}
\item Integer: \texttt{int}, used to represent integers.
\item Boolean: \texttt{bool}, used to represent boolean values.
\end{itemize}

\subsection{Structured Types}
Structured types hold groups of elements. There are three structured types in
SMURF: tuples, lists and functions.

Tuples have the format of 
\begin{verbatim}
(a, ..., n)
\end{verbatim}
where items \texttt{a - n} are elements in the tuple. Elements
of tuples can have different types.

Lists have the format of 
\begin{verbatim}
[a, ..., n]
\end{verbatim}
where items \texttt{a - n} are elements in the list. Elements
of lists must have the same type.

Functions have the format of 
\begin{verbatim}
arg1 -> arg2 -> ... -> argk -> return-value
\end{verbatim}
where \texttt{arg1 - argk} are the types of arguments of function.


\subsection{Derived Types}
SMURF has a derived type of \texttt{note}, which has the format of 
\begin{verbatim}
(pc, reg)^k[.]*
\end{verbatim}
where \texttt{pc} is an integer in the range from -1 to 11. When \texttt{pc} 
has a value of -1, the note is a rest, other it represents the pitch class of 
the note. 
\texttt{reg} is an integer in the range of 0-3, representing the register of the 
note. The register of the note is Bass 1, Bass 2, Treble 1 and Treble 2 for the
\texttt{reg} value from 0 to 3 repectively.
\texttt{k} is an integer of the power of 2, ranging from 1 to 16. 
Periods following \texttt{k} are optional. Users can add dots until the added duration
gets down to 16th note.

\begin{comment}

\subsubsection{Pitch}
pc (pitch classes) are represented by integers ranging from 0 to 11.
\begin{itemize}
  \item A Note with pc = -1 represents a rest. In this special case, the register for the Note only matters in relation
  to whether the rest lies on the treble or bass clef (i.e. whether the register is positive or negative)
\end{itemize}

\subsubsection{Beat}
A Beat represents a length of musical time. It has a Time tag and integer type. 
\begin{itemize}
  \item Must have the string ``Time" followed by an integer that is a power of 2 and \textless\space 32 in declaration
  \begin{itemize}
    \item whole note: Time 1
    \item half note: Time 2
    \item quarter note: Time 4
    \item eighth note: Time 8
    \item sixteenth note: Time 16
    \item thirty-second note: Time 32
  \end{itemize}
  \item Uses + operator to combine Time but only adds two operands that contain the same integer; recursively 
  checks for Time operands that contain the same integers until only unequal Time integers are left
  \begin{itemize}
    \item Time 4 + Time 16 + Time 16 + Time 16 + Time 16
    \item Time 4 + Time 8 + Time 16 + Time 16
    \item Time 4 + Time 8 + Time 8 
    \item Time 4 + Time 4 = Time 2 (quarter note + quarter note = half note)
  \end{itemize}
\end{itemize}

\subsubsection{Register}
Registers are represented by integers ranging from 0 to 3.
\begin{itemize}
  \item \begin{music}  \trebleclef  \end{music}  Treble Clef: notes middle C and
  higher represented by 2 and 3  
    \begin{itemize}
    \item middle C to the first B above middle C: 2 
    \item first C above middle C to next highest B: 3
    \end{itemize}
  \item \begin{music} \bassclef  \end{music}  Bass Clef: notes lower than middle
  C represented by 0 and 1 
    \begin{itemize}
    \item B directly below middle C to first C below middle C: 0
    \item next lowest B to next lowest C: 1
    \end{itemize}
\end{itemize}

\subsubsection{Note}
A Note is a tuple of three integers and is declared as 
\begin{verbatim}
(pc: int, beat: Beat, register: int)
\end{verbatim}

\subsubsection{Chord}
A Chord is a list of notes and is declared as [Note]. The compiler will check that all notes in the list have 
the same beat count.
\end{comment}

% \subsubsection{Measure} 
% Measure are abandoned in lrm



\section{Declarations and Bindings}

This section of the LRM describes the syntax and informal semantics of
declarations in SMURF. A program in SMURF, at its top-most level, is a
series of declarations. However, declarations may also occur inside of
\texttt{let} expressions. The scoping of such declarations is described 
in this sections. There are three types of declarations in SMURF: 
type signatures, pattern declarations, and function declarations.

\subsection{Type Signatures}

\begin{grammar}

<type-sig> $\rightarrow$ <identifier> :: <type>

<type> $\rightarrow$ Int | Bool | Note | [<type>] | 
										( <type>, $\ldots$, <type> ) | <type> -> <type> | ( <type> )
										

\end{grammar}

A type signature explicitly defines a type for a given identifier. The
\texttt{::} operator can be read as ``has type of." Only one type signature
for a given identifier can exist in a given scope. That is, two different
type signatures for a given identifier can exist, but they must be declared
in different scopes. There are three categories of types in SMURF: primitive
types,  structured types, and type synonyms. By convention, type
names are identifiers starting with an uppercase letter.

\subsubsection{Primitive Types}

The three types \texttt{Int}, \texttt{Bool}, and \texttt{Note} are
the fundamental building blocks of the type system in SMURF. 

\subsubsection{Structured Types}

SMURF has three structured types: lists, tuples, and functions. Each
type is represented by a special syntactic construct that operates on
other types to generate a concrete structured result.

The list type is written as \texttt{[t]} which specifies the type of lists
containing elements of type t.

The tuple type is written as \texttt{$(t_1, t_2, \ldots, t_n)$} where $t_i$
can be any type. THis specifies the type of tuples of size $n$ whose first
element has type $t_1$ second element has type $t_2$, and so on. A tuple
type must have at least two elements.

The function type is written as \texttt[$t_1 -> t_2$] and specifies the type
of functions that take an argument of type $t_1$ and return a value of type
$t_2$. As with tuple types, $t_1$ and $t_2$ do not have to be the same.
The function arrow is right-associative, so \texttt[Int -> Bool -> Bool] is
equivalent semantically to \texttt[Int -> (Bool -> Int)]

\subsubsection{Type Synonyms}

Type synonyms give different names to specific types, making our language
more readable and less verbose.

The \texttt[Chord] type is equivalent to the \texttt[$[$Note$]$] type.

The \texttt[System] type is equivalent to the \texttt[$[$Chord$]$] type.

\subsection{Pattern Declarations}

\begin{grammar}

<identifier> $\rightarrow$ <expr> | <guards>

<guards> $\rightarrow$ $|$ <bool-expr> $=$ <expr> $\\n$ <guards> |
											 $|$	otherwise $=$ <expr>
\end{grammar}

\subsection{Function Declarations}

\subsection{\texttt{main} Declaration}

\section{Expressions}


\subsection{Curried Applications}

    \subsubsection{Function declaration}
    The syntax of a function declaration is as follows: 
    \begin{verbatim} 
    function-expression :: argument-type-list -> result-type 
    \end{verbatim} 
    where
    \begin{verbatim}
    argument-type-list:     argument-type
                            argument-type-list  argument-type
    \end{verbatim} 
A function declaration must be on its own line and must declare a type. Declaring a general type is allowed. There are no explicit return statements.
  
    \subsubsection{Function application}
    The syntax of a function application is as follows: 
    \begin{verbatim}
    function-expression  argument-expression-list \end{verbatim} 
    where
    \begin{verbatim}
    argument-expression-list:     argument-expression
                                  argument-expression-list  argument-expression
    \end{verbatim} 
A function application associates from left to right, so parentheses are optional: 
    \begin{verbatim}
    funct a b
    \end{verbatim}
    is equivalent to
    \begin{verbatim}
    ((funct a) b) 
    \end{verbatim}
    Parentheses are used to change the precedence from the default. The following evaluates funct1 with argument b then evaluates funct2 with argument a:
    \begin{verbatim}
    funct2 a (funct1 b)
    \end{verbatim}   

  \subsubsection{**PARTIAL APPLICATION**}
  

\subsection{Operator Application}
  The syntax for applying a binary operator to two expressions is infix:
    \begin{verbatim}
    expression  operator  expression \end{verbatim} 
    where
    \begin{verbatim}
    operator:     arithmetic-operator
                  comparison-operator
                  boolean-operator
                  list-operator
                  function-operator \end{verbatim} 

\subsection{Conditionals}
  The syntax for conditional expressions is as follows:
  \begin{verbatim}
  if  expression  then expression-true  else expression-false \end{verbatim} 
  When the value of expression evaluates to true, expression-true is evaluated, otherwise expression-false is evaluated. Conditional expressions do not have newline restrictions.

\subsection{Lists}
Lists are written as:
  \begin{verbatim}
  [expression-list]\end{verbatim}
  where
  \begin{verbatim}
  expression-list:     <empty>
                       expression
                       expression-list,  expression \end{verbatim}
[expression$_{1}$, ..., expression$_{k}$]  =  expression$_{1}$ : ( expression$_{2}$ : (... (expression$_{k}$ : [ ] )) \\ \\
where \textit{k} \textgreater\space 0. The expressions in a list must all be of the same type. Both the list constructor : and empty list [ ] are reserved as part of the language syntax and therefore cannot be hidden or redefined. The list constructor has right associativity.

\subsection{Tuples}
Tuples are written as:
  \begin{verbatim}
  (expression-list)\end{verbatim}
  where
  \begin{verbatim}
  expression-list:     expression, expression
                       expression-list,  expression \end{verbatim}
The expressions in a tuple may be of different types. The constructor of an n-tuple is denoted by (\textunderscore
, ..., \textunderscore) where there are \textit{n-1} commas.

\subsection{Parenthesized Expressions}
Parenthesized expressions has the form:
  \begin{verbatim}
  (expression)\end{verbatim}
  where expression is evaluated as a primary expression.

\subsection{Expression Type-Signature}
Expression type-signatures have the form:
  \begin{verbatim}
  expression :: type \end{verbatim}
  where expression is an expression and type is a type. This is used to explicitly define a type for an expression. The declared type may be more specific than the principal type but it is illegal to give a declared type that is more general than the principal type. Giving a declared type that is not comparable to the principal type will also yield an error.

\subsection{Let Expressions}
\begin{grammar}
<let-exp> $\rightarrow$ let <decls> in <expression>
\end{grammar}
Let expressions have the form \emph{let \{} $d_1;...;d_n$ \} \emph{in e}, where $d_n$ is the nth declaration and \emph{e} is an expression. Note that the scope of $d_n$ is in \emph{e}.

% are pattern bindings matched lazily?

\subsection{Pattern Matching}

Patterns can be found in function definitions, pattern bindings, and list operations. Patterns are matched with values. Matching a pattern can either be successful or it can fail. A successful match will return  a binding for the variables in the pattern. 

\subsection{Guards}
\begin{grammar}
<guard> $\rightarrow$  let <decls> | <infixexp>          
\end{grammar}

There are two kinds of guards in SMURF: local bindings and boolean guards. Local bindings have the form let $<decls>$ and introduces the declarations to the program environment. Boolean guards are expressions of type Bool. The boolean guard succeeds if the expression evaluates to True.


\section{Declarations and Bindings}

This section of the LRM describes the syntax and informal semantics of
declarations in SMURF. A program in SMURF, at its top-most level, is a
series of declarations. However, declarations may also occur inside of
\texttt{let} expressions. The scoping of such declarations is described 
in this sections. There are three types of declarations in SMURF: 
type signatures, pattern declarations, and function declarations.

\subsection{Type Signatures}

\begin{grammar}

<type-sig> $\rightarrow$ <identifier> :: <type>

<type> $\rightarrow$ Int | Bool | Note | [<type>] | 
										( <type>, $\ldots$, <type> ) | <type> -> <type> | ( <type> )
										

\end{grammar}

A type signature explicitly defines a type for a given identifier. The
\texttt{::} operator can be read as ``has type of." Only one type signature
for a given identifier can exist in a given scope. That is, two different
type signatures for a given identifier can exist, but they must be declared
in different scopes. There are three categories of types in SMURF: primitive
types,  structured types, and type synonyms. By convention, type
names are identifiers starting with an uppercase letter.

\subsubsection{Primitive Types}

The three types \texttt{Int}, \texttt{Bool}, and \texttt{Note} are
the fundamental building blocks of the type system in SMURF. 

\subsubsection{Structured Types}

SMURF has three structured types: lists, tuples, and functions. Each
type is represented by a special syntactic construct that operates on
other types to generate a concrete structured result.

The list type is written as \texttt{[t]} which specifies the type of lists
containing elements of type t.

The tuple type is written as \texttt{$(t_1, t_2, \ldots, t_n)$} where $t_i$
can be any type. THis specifies the type of tuples of size $n$ whose first
element has type $t_1$ second element has type $t_2$, and so on. A tuple
type must have at least two elements.

The function type is written as \texttt[$t_1 -> t_2$] and specifies the type
of functions that take an argument of type $t_1$ and return a value of type
$t_2$. As with tuple types, $t_1$ and $t_2$ do not have to be the same.
The function arrow is right-associative, so \texttt[Int -> Bool -> Bool] is
equivalent semantically to \texttt[Int -> (Bool -> Int)]

\subsubsection{Type Synonyms}

Type synonyms give different names to specific types, making our language
more readable and less verbose.

The \texttt[Chord] type is equivalent to the \texttt[$[$Note$]$] type.

The \texttt[System] type is equivalent to the \texttt[$[$Chord$]$] type.

\subsection{Pattern Declarations}

\begin{grammar}

<identifier> $\rightarrow$ <expr> | <guards>

<guards> $\rightarrow$ $|$ <bool-expr> $=$ <expr> $\\n$ <guards> |
											 $|$	otherwise $=$ <expr>
\end{grammar}

\subsection{Function Declarations}

\subsection{\texttt{main} Declaration}

\section{Library Functions}

Below are the library functions that can be used in the SMURF language. 
These functions are implemented in SMURF but are available to all users 
for their convenience. Each library function will include a description 
and its SMURF definition. \newline

\noindent\textbf{Head}

The function \texttt{head} takes a list and returns the first element. 
This function is commonly used when working with lists. The head function
is available for all lists through the use of polymorphic typing. 

\begin{verbatim}
head_note :: [a] -> a
head (h:tl) = h
\end{verbatim} 


\noindent\textbf{Tail}

The function \texttt{tail} takes a list and returns the end of the list.
This function is commonly used when working with lists. 

\begin{verbatim}
tail_note :: [a] -> [a]
tail (h:tl) = tl
\end{verbatim} 


\noindent\textbf{MakeNotes}

The function \texttt{makeNotes} takes in three lists and returns a list 
of notes. The first list consists of expressions of type \texttt{Int} representing 
pitches or rests. The second list consists of expressions of type \texttt{Int}
representing the register that the pitch will be played in. The third list is a list 
of expressions of type \texttt{Beat} representing a set of durations. It is common in 
12 tone serialism to use lists of notes. This function allows the user 
to easily turn their modified rows and columns into a list of notes to add 
to their system. 

\begin{verbatim}
makeNotes :: [Int] -> [Int] -> [Beat] -> [Note]
makeNotes [] = []
makeNotes (h1:tl1) (h2:tl2) (h3:tl3) = (h1,h2)$h3:(makeNotes tl1 tl2 tl3)
\end{verbatim}

\clearpage

% citations begin here
\bibliographystyle{ieeetr}
\bibliography{ref/refs}


\end{document}
