\documentclass[dvips, 12pt]{article}

% Any percent sign marks a comment to the end of the line

% Every latex document starts with a documentclass declaration like this
% The option dvips allows for graphics, 12pt is the font size, and article
%   is the style


\usepackage[margin=1in]{geometry}

\usepackage{float}
\restylefloat{table}

\usepackage{tabularx}
\usepackage{graphicx}
\usepackage{url}
\usepackage{harmony}
\usepackage{musixtex}
\usepackage{subfigure}
\usepackage{caption}
\usepackage{listings}
\lstset{language=Haskell}
\usepackage{syntax}
\usepackage{comment}


\usepackage{color}
%\usepackage{biblatex}
%\usepackage{subcaption}

% These are additional packages for "pdflatex", graphics, and to include
% hyperlinks inside a document.


% These force using more of the margins that is the default style


% Everything after this becomes content
% Replace the text between curly brackets with your own

\title{{\Huge \bfseries SMURF Language Reference Manual} \\ \Large \it Serial MUsic Represented as Functions \vspace{0.6cm}}

\author{\normalsize Richard Townsend, Lianne Lairmore, Lindsay Neubauer, Van Bui, Kuangya Zhai
	\\ \small \{rt2515, lel2143, lan2135, vb2363, kz2219\}@columbia.edu \vspace{0.6cm}}

\date{\today \vspace{2cm}}

% You can leave out "date" and it will be added automatically for today
% You can change the "\today" date to any text you like

\begin{document}
\maketitle
\clearpage


% This command causes the title to be created in the document
\tableofcontents

\section{Syntax Notation}
The syntax notation used in this manual are as follows. Syntactic 
categories are indicated by \emph{italic} type. Literal words and 
characters will be displayed in \texttt{typeset}. Alternatives are listed 
on separate lines. Optional terminal and non-terminals are indicated by the
subscript\_{opt}. 

Regular expression notations are used to specigy grammar patterns in this manual.
r* means the pattern r may appear zero or more time, r+ means the r may appear one or more 
times, r? means r may appear zero or once. r1 | r2 denotes an option between two patterns, r1 r2 denotes 
r1 followed by r2. 


\section{Lexical Conventions}
SMURF programs are lexically composed of three elements: comments, tokens, and whitespace.

\subsection{Comments}
SMURF allows nested, multiline comments in addition to single line comments.
\begin{table} [H]
\centering
\begin{tabularx}{\textwidth}{lXl}
\hline\hline
Comment Symbols & Description & Example \\
\hline\hline
  \texttt{/* */} & Multiline comments, nesting allowed & \texttt{/* This /* is all */ commented */} \\ \hline
  \texttt{//} & Single-line comment & \texttt{// This is a comment} \\ \hline
\end{tabularx}
\end{table}


\subsection{Tokens}
In SMURF, a token is a string of one or more characters that is significant as a group.
SMURF has 6 kinds of tokens: {\it identifiers}, {\it keywords}, {\it constants},
      {\it operators},
{\it separators} and {\it newlines}.

\subsubsection{Identifiers}
\label{sec:identifiers}
An identifier consists of a letter followed by other letters, 
digits and underscores. The letters are the ASCII characters \texttt{a}-\texttt{z} and
\texttt{A}-\texttt{Z}. Digits are ASCII characters \texttt{0}-\texttt{9}. SMURF is case sensitive.

\begin{grammar}
<letter> $\rightarrow$ [`a'-`z' `A'-`Z'] 

<digit> $\rightarrow$ [`0'-`9'] 

<underscore> $\rightarrow$ {`_'} 

<identifier> $\rightarrow$ <letter> (<letter> | <digit> | <underscore>)*
\end{grammar}

\subsubsection{Keywords}
\label{sec:keywords}
Keywords in SMURF are identifiers reserved by the language. Thus, they are not available for
re-definition or overloading by users. 

\begin{table} [H]
	\centering
    \begin{tabular}{ll}
    \hline\hline
    Keywords & Descriptions \\ 
    \hline\hline
      \texttt{Bool} & Boolean data type \\ \hline
      \texttt{Int} & Integer data type \\ \hline
      \texttt{Note} & Atomic musical data type \\ \hline
      \texttt{Beat} & Note duration data type\\ \hline
      \texttt{Chord} & Data type equivalent to \texttt{[Note]} type \\ \hline
      \texttt{System} & Data type equivalent to \texttt{[Chord]} type \\ \hline
      \texttt{True, False} & Boolean constants \\ \hline
      \texttt{let, in} & Allow local bindings in expressions  \\ \hline
      \texttt{if, then, else} & Specify conditional expression, else compulsory  \\ \hline
      \texttt{random} & Generate random numbers \\ \hline
      \texttt{print} & Print information to standard output \\ \hline
      \texttt{main} & Specify the value of a SMURF program\\ \hline
    \end{tabular}
\end{table}


\subsubsection{Constants}
\label{sec:constants}
In SMURF, constants are expressions with a fixed value. Integer literals and
Boolean keywords are the constants of SMURF. 

\setlength{\grammarindent}{6em}
\begin{grammar}
<digit> $\rightarrow$ [`0'-`9'] 

<constant> $\rightarrow$ \texttt{-}? [`1'-`9'] <digit>* \\
												 \texttt{0} <digit>* \\
												\texttt{True} \\
												\texttt{False}
\end{grammar}

\subsubsection{Operators}
SMURF permits arithmetic, comparison, boolean, list, declaration, and row operations, all of which
are carried out through the use of specific operators. The syntax and semantics of all of these
operators are described in sections \ref{sec:postfixop}, \ref{sec:prefixop}, and \ref{sec:binaryop},
except for declaration operators, which are described in section \ref{sec:declarations}.


\subsubsection{Newlines}
SMURF uses newlines to signify the end of a declaration, except
when preceded by the \texttt{\textbackslash} token. In the latter case, the newline is ignored by the compiler 
(see example below). If no such token precedes a newline, then the compiler will treat the newline as
a token being used to terminate a declaration.

\subsubsection{Separators}

\begin{grammar}
<separator> $\rightarrow$ \texttt{,} \\
												  \texttt{\&} \\
													\texttt{\textbackslash}
\end{grammar}

Separators in SMURF are special tokens used to separate other tokens. 
Commas are used to separate elements in a list.
The \texttt{\&} symbol can be used in place of a newline. That is, the compiler
will replace all \texttt{\&} characters with newlines. The
\texttt{\textbackslash} token, when followed by a newline token,
may be used to splice two lines. E.g.
\begin{lstlisting}
genAltChords (x:y:ys) = [(x,Time 4,1)]   \
                        :[(y,Time 4,-1)]:genAltChords ys
\end{lstlisting}
is the same as 
\begin{lstlisting}
genAltChords (x:y:ys) = [(x,Time 4,1)]:[(y,Time 4,-1)]:genAltChords ys
\end{lstlisting}


\subsection{Whitespace}
\label{sec:whitespaces}
{\it Whitespace} consists of any sequence of {\it blank} and {\it tab} characters.
Whitespace is used to
separate tokens and format programs. All whitespace is ignored by the
SMURF compiler. As a result, indentations are not significant in SMURF.


\section{Meaning of Identifiers}
In SMURF, an identifier is a name for a variable or function. The naming rule of
identifier is defined in section~\ref{sec:identifiers}. Keywords defined in
section~\ref{sec:keywords} are not available for the use of identifiers.


\subsection{Purpose}
\subsubsection{Functions}
Functions in SMURF enable users to structure programs in a more modular way. 
Each function has input and output, whose types need to be explicitly defined by
users. The function describles how to produce the output from its input.
SMURF is a side effect free language, which means
that if provided with the same input, a function is guaranteed to give the same
output. 


\subsubsection{variables}
In SMURF, a variable is an identifier that is linked to a value stored in the
system's memory or an expression that can be evaluated.
A variable is an abstraction of a computer memory cell or a collection
of memory cells. 
SMURF is a strongly typed programming language, which means the type of a variable can't
be changed once declared. Each variable has a static type which can be automatically
deduced by the SMURF compiler, or explicitly defined by users. The variables in SMURF are immutable.


\subsection{Scope and Lifetime}
In SMURF, a variable is bound in its scope to a value using constructs like
\texttt{let}. A variable is visible within its scope.
There is no global variables in SMURF. A variable becomes invalid after the 
ending of its scope. E.g.

\begin{lstlisting}
let prime = [2,0,4,6,8,10,1,3,5,7,9,11] 
in let prime = [0,2,4,6,8,10,1,3,5,7,9,11]
       p3 = trans 3 prime 
   in print (head p3)
\end{lstlisting}

where {\texttt prime} and {\texttt p3} are bound by the {\texttt let}
expression. 
In line 2, {\texttt prime} is
redefined in a inner {\texttt let} expression. In line 3, the {\texttt
trans} function sees the newly defined {\texttt prime}. So the result
to be printed in line 4 should be 3 (0+3). After the finishing of line
4, all the variables defined in this example will be invalid and can't be
accessed anymore.

\subsection{Basic Types}
There are three fundamental types in SMURF: {\texttt Int}, {\texttt Bool} and {\texttt Beat}. 
\begin{itemize}
\item {\it Integer}: \texttt{Int}, used to represent integers.
\item {\it Boolean}: \texttt{Bool}, used to represent boolean values.
\item {\it Beat}   : \texttt{Beat}, used to represent the duration of a note. The value of {\texttt Beat} are integers of the power of 2, ranging from 1 to 16.
\end{itemize}

\subsection{Structured Types}
Structured types hold groups of elements. There are two structured types in
SMURF: {\it list} and {\it function}.

{\it list} has the format of 
\begin{grammar}
<list> $\rightarrow$ [{\it a, ..., n}]
\end{grammar}
where items {\it a - n} are elements in the list. Elements
of lists must have the same type.

{\it function} has the format of 
\begin{grammar}
<function> $\rightarrow$ {\it arg1} {\texttt -\textgreater} ... {\texttt
    -\textgreater} {\it argk}
{\texttt -\textgreater} {\it return-value}
\end{grammar}
where {\it arg1 - argk} are the input of function and {\it return-value} is the
output of the function.


\subsection{Derived Types}
Besides the basic types, SMURF has also several derived types. 

The type of \texttt{Note} is used to represent a musical note in SMURF. The format of \texttt{Note} is defined as below.
\begin{grammar}
<note> $\rightarrow$ ({\it pc}, {\it reg})\^{\it k}[.]* 
\end{grammar}
where {\it pc} is an integer in the range from -1 to 11. When {\it pc} 
has a value of -1, the note is a rest, otherwise it represents the pitch class of 
the note. 
{\it reg} is an integer in the range of 0-3, representing the register of the 
note, where the integer values and corresponding registers are given below.
\begin{itemize}
    \item 0: Bass clef, B directly below middle C to first C below middle C
    \item 1: Bass clef, next lowest B to next lowest C
    \item 2: Treble clef, middle C to the first B above middle C
    \item 3: Treble clef, first C above middle C to next highest B
\end{itemize}
{\it k} stands for the beat of the note.
Periods following {\it k} are optional. Users can add dots until the added duration
gets down to 16th note.

The type of \texttt{Chord} is used to represent several notes to be played simultaneously. It is defined as 
\begin{grammar}
<chord> $\rightarrow$ [{\it note}]
\end{grammar}
The compiler will check to make sure all the {\it note} in the same {\it chord} have the same time duration.

The type of \texttt{System} is used to represent a list of {\it chord} to be played in a sequential patterns. It is defined as 
\begin{grammar}
<system> $\rightarrow$ [{\it chord}]
\end{grammar}


\begin{comment}

\subsubsection{Pitch}
pc (pitch classes) are represented by integers ranging from 0 to 11.
\begin{itemize}
  \item A Note with pc = -1 represents a rest. In this special case, the register for the Note only matters in relation
  to whether the rest lies on the treble or bass clef (i.e. whether the register is positive or negative)
\end{itemize}

\subsubsection{Beat}
A Beat represents a length of musical time. It has a Time tag and integer type. 
\begin{itemize}
  \item Must have the string ``Time" followed by an integer that is a power of 2 and \textless\space 32 in declaration
  \begin{itemize}
    \item whole note: Time 1
    \item half note: Time 2
    \item quarter note: Time 4
    \item eighth note: Time 8
    \item sixteenth note: Time 16
    \item thirty-second note: Time 32
  \end{itemize}
  \item Uses + operator to combine Time but only adds two operands that contain the same integer; recursively 
  checks for Time operands that contain the same integers until only unequal Time integers are left
  \begin{itemize}
    \item Time 4 + Time 16 + Time 16 + Time 16 + Time 16
    \item Time 4 + Time 8 + Time 16 + Time 16
    \item Time 4 + Time 8 + Time 8 
    \item Time 4 + Time 4 = Time 2 (quarter note + quarter note = half note)
  \end{itemize}
\end{itemize}

\subsubsection{Register}
Registers are represented by integers ranging from 0 to 3.
\begin{itemize}
  \item \begin{music}  \trebleclef  \end{music}  Treble Clef: notes middle C and
  higher represented by 2 and 3  
    \begin{itemize}
    \item middle C to the first B above middle C: 2 
    \item first C above middle C to next highest B: 3
    \end{itemize}
  \item \begin{music} \bassclef  \end{music}  Bass Clef: notes lower than middle
  C represented by 0 and 1 
    \begin{itemize}
    \item B directly below middle C to first C below middle C: 0
    \item next lowest B to next lowest C: 1
    \end{itemize}
\end{itemize}

\subsubsection{Note}
A Note is a tuple of three integers and is declared as 
\begin{verbatim}
(pc: int, beat: Beat, register: int)
\end{verbatim}

\subsubsection{Chord}
A Chord is a list of notes and is declared as [Note]. The compiler will check that all notes in the list have 
the same beat count.
\end{comment}

% \subsubsection{Measure} 
% Measure are abandoned in lrm



\section{Declarations and Bindings}
\label{sec:declarations}
\setlength{\grammarindent}{4em}

This section of the LRM describes the syntax and informal semantics of
declarations in SMURF. Declarations may occur at a global level of scope
in a SMURF program or inside of
\texttt{let} expressions. The scoping of such declarations is described 
in this section. There are three types of declarations in SMURF: 
type signatures, definitions, and function declarations.

\subsection{Type Signatures}

\begin{grammar}

<type-sig> $\rightarrow$ <identifier> \texttt{::} <type>

<type> $\rightarrow$ \texttt{Int} \\ \texttt{Bool} \\ \texttt{Beat} \\ \texttt{Note} \\
											\texttt{Chord} \\ \texttt{System} \\ <identifier> \\ 
											\texttt{[} <type> \texttt{]} \\ \texttt{(} <type>\texttt{,} 
											$\ldots$\texttt{,} <type> \texttt{)} \\ <type> \texttt{->} <type>
											 \\ \texttt{(} <type> \texttt{)}
										

\end{grammar}

A type signature explicitly defines the type for a given identifier. The
\texttt{::} operator can be read as ``has type of." Only one type signature
for a given identifier can exist in a given scope. That is, two different
type signatures for a given identifier can exist, but they must be declared
in different scopes. There are three categories of types in SMURF: basic,  structured, and derived types; types are described in sections~\ref{sec:basictype}-\ref{sec:derivedtype}.

% type synonyms? 

SMURF also permits the use of polymorphism in specifying the types of identifiers. If some
non-keyword identifier appears on the right-hand side of the \texttt{::} operator, it means
that that identifier could be replaced with any standard SMURF type. For example, if we have
the type signature \texttt{x :: a -> b}, we have defined a function \texttt{x} that takes
an argument of any type and returns an argument of any type. That is, \texttt{x} could take
an argument of type \texttt{Int} and return a value of type \texttt{Char}.
If the same identifier is used multiple times in the same type signature, 
it must take on the same type everywhere that it appears. If our last example
was instead defined \texttt{x :: a -> a}, we would have defined a function that took an argument
of any type and returned a value of the \emph{same} type. That is, \texttt{x} could take an argument
of type \texttt{Int}, but then it would be forced to return a value of type \texttt{Int} as well.

%\subsubsection{Primitive Types}
%
%The four types \texttt{Int}, \texttt{Bool}, \texttt{Beat}, and \texttt{Note} are
%the fundamental building blocks of the type system in SMURF. \texttt{Int} and \texttt{Bool}
%are the traditional integer and boolean types. The \texttt{Beat} type is a subset of 
%\texttt{Int}, in that a value of type \texttt{Int} can be specified as type \texttt{Beat} if
%it is a power of two between 1 and 16. 
%The note type is written as \texttt{(Int, Int)\string^Beat}. The first expression of type
%\texttt{Int} is a pitch class, the next is a register, and the expression of type \texttt{Beat}
%is the durational value of the note.
%
%\subsubsection{Structured Types}
%
%SMURF has four structured types: lists, tuples, functions, and notes. Each
%type is represented by a special syntactic construct that operates on
%other types to generate a concrete structured result.
%
%The list type is written as \texttt{[} $t$ \texttt{]} which specifies the type of lists
%containing elements of type $t$.
%
%The tuple type is written as \texttt{($t_1$, $t_2$, \ldots, $t_n$)} where $t_i$
%can be any type. This specifies the type of tuples of size $n$ whose first
%element has type $t_1$, second element has type $t_2$, and so on. A tuple
%type must have at least two elements.
%
%The function type is written as $t_1$ \texttt{->} $t_2$ and specifies the type
%of functions that take an argument of type $t_1$ and return a value of type
%$t_2$. As with tuple types, $t_1$ and $t_2$ do not have to be the same.
%The function arrow is right-associative, so \texttt{Int -> Bool -> Bool} is
%equivalent semantically to \texttt{Int -> (Bool -> Int)}
%
%
%\subsubsection{Type Synonyms}
%
%Type synonyms give different names to specific types, making our language
%more readable and less verbose.
%The \texttt{Chord} type is equivalent to the \texttt{[Note]} type.
%The \texttt{System} type is equivalent to the \texttt{[Chord]} type.

\subsection{Definitions}

\begin{grammar}

<definition> $\rightarrow$ <identifier> $=$ <expression>

\end{grammar}

A definition binds an identifier to an expression. All definitions at a given
scope must be unique and can be mutually recursive. For example, the following
is legal in SMURF:

\begin{verbatim}
let x = 4
  	z = if y == 7 then x else y
  	y = let x = 5 
        in x + 3 
in x + z + y
\end{verbatim}


The $x$ in the nested let statement is in a different scope than the $x$ in the global let statement, 
so the two definitions do not conflict. $z$ is able to refer to $y$ even though
$y$ is defined after $z$ in the program. In this example, the three 
identifiers $x, y, $ and $z$ in the global \texttt{let} will evaluate to values 4, 8, and 8, respectively,
while the identifier $x$ after the nested let statement will evaluate to 5.

A type signature may be given for a definition but is not required.

\subsection{Function Declarations}

\begin{grammar}

<fun-dec> $\rightarrow$ <identifier> <args> $=$ <expression> \\
												 <func-dec> <new-line> <func-dec>

<args> $\rightarrow$ <pattern> \\ <pattern> <args>
\end{grammar}

A function declaration defines an identifier as a function that takes
some number of patterns as arguments and, based on which patterns are matched
when the function is called, returns the result
of a given expression. Essentially, a function declaration can be seen as a 
sequence of one or more assignment expressions, where each assignment defines
a value bound to the function identifier. The actual value returned by the function
is dependent on the patterns matched in the function call. There must be at least one pattern listed
as an argument in a function declaration. All assignment expressions defining
a function must be contiguous and separated by new lines, and the number of
patterns as arguments in each assignment expression must be the same.

As with definitions, only one declaration of a given function name can exist
in a given scope. However, the same function name can be declared multiple
times if each instance is in a different scope.

If a function declaration for some identifier $x$ occurs in scope $n$, then
a type signature for $x$ in scope $k>n$ is required. That is if a function has
been declared but its type has not been explicitly stated in the same or a higher
scope, a compile-time error will be generated. The type of the arguments
passed to a function are checked at compile-time as well, and an error
is issued if they don't match the types specified in that function's 
type signature.

\subsection{\texttt{main} Declaration}

Every SMURF program must define the reserved identifier \texttt{main}. This
identifier may only be used on the left-hand side of a top-level definition. The expression
bound to \texttt{main} is evaluated and its value is the value of the SMURF program itself.
That is, when a SMURF program is compiled and run, the expression bound to \texttt{main} is
evaluated and the result is converted to our bytecode representation of a MIDI file. As a result,
this expression must evaluate to a value of type \texttt{[]}, \texttt{Note}, \texttt{Chord}, 
\texttt{System}, or \texttt{[System]}. If a definition for \texttt{main} is not included in a 
SMURF program or if the expression bound to it does not have one of the types outlined above,
a compile-time error will occur.


\section{Expressions}

\subsection{Description of Precedence}


\subsection{Primary Expressions}

\subsubsection{Identifiers}
An identifier expression is an expression that involves only identifiers. Its 
type is specified at definition. Identifiers in SMURF are defined in 
section~\ref{sec:identifiers}.

\subsubsection{Constants}
A constant expression is an expression that involves only constants. Constants 
in SMURF are defined in section~\ref{sec:constants}.

\subsubsection{(Expression)}
An expression surrounded by parentheses is a new expression.


\subsection{Curried Applications}

    \subsubsection{Function declaration}
    The syntax of a function declaration is as follows: 
    \begin{verbatim} 
    function-expression :: argument-type-list -> result-type 
    \end{verbatim} 
    where
    \begin{verbatim}
    argument-type-list:     argument-type
                            argument-type-list  argument-type
    \end{verbatim} 
A function declaration must be on its own line and must declare a type. Declaring a general type is allowed. There are no explicit return statements.
  
    \subsubsection{Function application}
    The syntax of a function application is as follows: 
    \begin{verbatim}
    function-expression  argument-expression-list \end{verbatim} 
    where
    \begin{verbatim}
    argument-expression-list:     argument-expression
                                  argument-expression-list  argument-expression
    \end{verbatim} 
A function application associates from left to right, so parentheses are optional: 
    \begin{verbatim}
    funct a b
    \end{verbatim}
    is equivalent to
    \begin{verbatim}
    ((funct a) b) 
    \end{verbatim}
    Parentheses are used to change the precedence from the default. The following evaluates funct1 with argument b then evaluates funct2 with argument a:
    \begin{verbatim}
    funct2 a (funct1 b)
    \end{verbatim}   

  \subsubsection{**PARTIAL APPLICATION**}
  

\subsection{Operator Application}
  The syntax for applying a binary operator to two expressions is infix:
    \begin{verbatim}
    expression  operator  expression \end{verbatim} 
    where
    \begin{verbatim}
    operator:     arithmetic-operator
                  comparison-operator
                  boolean-operator
                  list-operator
                  function-operator \end{verbatim} 

\subsection{Conditionals}
  The syntax for conditional expressions is as follows:
  \begin{verbatim}
  if  expression  then expression-true  else expression-false \end{verbatim} 
  When the value of expression evaluates to true, expression-true is evaluated, otherwise expression-false is evaluated. Conditional expressions do not have newline restrictions.

\subsection{Lists}
Lists are written as:
  \begin{verbatim}
  [expression-list]\end{verbatim}
  where
  \begin{verbatim}
  expression-list:     <empty>
                       expression
                       expression-list,  expression \end{verbatim}
[expression$_{1}$, ..., expression$_{k}$]  =  expression$_{1}$ : ( expression$_{2}$ : (... (expression$_{k}$ : [ ] )) \\ \\
where \textit{k} \textgreater\space 0. The expressions in a list must all be of the same type. Both the list constructor : and empty list [ ] are reserved as part of the language syntax and therefore cannot be hidden or redefined. The list constructor has right associativity.

\subsection{Tuples}
Tuples are written as:
  \begin{verbatim}
  (expression-list)\end{verbatim}
  where
  \begin{verbatim}
  expression-list:     expression, expression
                       expression-list,  expression \end{verbatim}
The expressions in a tuple may be of different types. The constructor of an n-tuple is denoted by (\textunderscore
, ..., \textunderscore) where there are \textit{n-1} commas.

\subsection{Parenthesized Expressions}
Parenthesized expressions has the form:
  \begin{verbatim}
  (expression)\end{verbatim}
  where expression is evaluated as a primary expression.

\subsection{Expression Type-Signature}
Expression type-signatures have the form:
  \begin{verbatim}
  expression :: type \end{verbatim}
  where expression is an expression and type is a type. This is used to explicitly define a type for an expression. The declared type may be more specific than the principal type but it is illegal to give a declared type that is more general than the principal type. Giving a declared type that is not comparable to the principal type will also yield an error.

\section{Declarations and Bindings}
\label{sec:declarations}
\setlength{\grammarindent}{4em}

This section of the LRM describes the syntax and informal semantics of
declarations in SMURF. Declarations may occur at a global level of scope
in a SMURF program or inside of
\texttt{let} expressions. The scoping of such declarations is described 
in this section. There are three types of declarations in SMURF: 
type signatures, definitions, and function declarations.

\subsection{Type Signatures}

\begin{grammar}

<type-sig> $\rightarrow$ <identifier> \texttt{::} <type>

<type> $\rightarrow$ \texttt{Int} \\ \texttt{Bool} \\ \texttt{Beat} \\ \texttt{Note} \\
											\texttt{Chord} \\ \texttt{System} \\ <identifier> \\ 
											\texttt{[} <type> \texttt{]} \\ \texttt{(} <type>\texttt{,} 
											$\ldots$\texttt{,} <type> \texttt{)} \\ <type> \texttt{->} <type>
											 \\ \texttt{(} <type> \texttt{)}
										

\end{grammar}

A type signature explicitly defines the type for a given identifier. The
\texttt{::} operator can be read as ``has type of." Only one type signature
for a given identifier can exist in a given scope. That is, two different
type signatures for a given identifier can exist, but they must be declared
in different scopes. There are three categories of types in SMURF: basic,  structured, and derived types; types are described in sections~\ref{sec:basictype}-\ref{sec:derivedtype}.

% type synonyms? 

SMURF also permits the use of polymorphism in specifying the types of identifiers. If some
non-keyword identifier appears on the right-hand side of the \texttt{::} operator, it means
that that identifier could be replaced with any standard SMURF type. For example, if we have
the type signature \texttt{x :: a -> b}, we have defined a function \texttt{x} that takes
an argument of any type and returns an argument of any type. That is, \texttt{x} could take
an argument of type \texttt{Int} and return a value of type \texttt{Char}.
If the same identifier is used multiple times in the same type signature, 
it must take on the same type everywhere that it appears. If our last example
was instead defined \texttt{x :: a -> a}, we would have defined a function that took an argument
of any type and returned a value of the \emph{same} type. That is, \texttt{x} could take an argument
of type \texttt{Int}, but then it would be forced to return a value of type \texttt{Int} as well.

%\subsubsection{Primitive Types}
%
%The four types \texttt{Int}, \texttt{Bool}, \texttt{Beat}, and \texttt{Note} are
%the fundamental building blocks of the type system in SMURF. \texttt{Int} and \texttt{Bool}
%are the traditional integer and boolean types. The \texttt{Beat} type is a subset of 
%\texttt{Int}, in that a value of type \texttt{Int} can be specified as type \texttt{Beat} if
%it is a power of two between 1 and 16. 
%The note type is written as \texttt{(Int, Int)\string^Beat}. The first expression of type
%\texttt{Int} is a pitch class, the next is a register, and the expression of type \texttt{Beat}
%is the durational value of the note.
%
%\subsubsection{Structured Types}
%
%SMURF has four structured types: lists, tuples, functions, and notes. Each
%type is represented by a special syntactic construct that operates on
%other types to generate a concrete structured result.
%
%The list type is written as \texttt{[} $t$ \texttt{]} which specifies the type of lists
%containing elements of type $t$.
%
%The tuple type is written as \texttt{($t_1$, $t_2$, \ldots, $t_n$)} where $t_i$
%can be any type. This specifies the type of tuples of size $n$ whose first
%element has type $t_1$, second element has type $t_2$, and so on. A tuple
%type must have at least two elements.
%
%The function type is written as $t_1$ \texttt{->} $t_2$ and specifies the type
%of functions that take an argument of type $t_1$ and return a value of type
%$t_2$. As with tuple types, $t_1$ and $t_2$ do not have to be the same.
%The function arrow is right-associative, so \texttt{Int -> Bool -> Bool} is
%equivalent semantically to \texttt{Int -> (Bool -> Int)}
%
%
%\subsubsection{Type Synonyms}
%
%Type synonyms give different names to specific types, making our language
%more readable and less verbose.
%The \texttt{Chord} type is equivalent to the \texttt{[Note]} type.
%The \texttt{System} type is equivalent to the \texttt{[Chord]} type.

\subsection{Definitions}

\begin{grammar}

<definition> $\rightarrow$ <identifier> $=$ <expression>

\end{grammar}

A definition binds an identifier to an expression. All definitions at a given
scope must be unique and can be mutually recursive. For example, the following
is legal in SMURF:

\begin{verbatim}
let x = 4
  	z = if y == 7 then x else y
  	y = let x = 5 
        in x + 3 
in x + z + y
\end{verbatim}


The $x$ in the nested let statement is in a different scope than the $x$ in the global let statement, 
so the two definitions do not conflict. $z$ is able to refer to $y$ even though
$y$ is defined after $z$ in the program. In this example, the three 
identifiers $x, y, $ and $z$ in the global \texttt{let} will evaluate to values 4, 8, and 8, respectively,
while the identifier $x$ after the nested let statement will evaluate to 5.

A type signature may be given for a definition but is not required.

\subsection{Function Declarations}

\begin{grammar}

<fun-dec> $\rightarrow$ <identifier> <args> $=$ <expression> \\
												 <func-dec> <new-line> <func-dec>

<args> $\rightarrow$ <pattern> \\ <pattern> <args>
\end{grammar}

A function declaration defines an identifier as a function that takes
some number of patterns as arguments and, based on which patterns are matched
when the function is called, returns the result
of a given expression. Essentially, a function declaration can be seen as a 
sequence of one or more assignment expressions, where each assignment defines
a value bound to the function identifier. The actual value returned by the function
is dependent on the patterns matched in the function call. There must be at least one pattern listed
as an argument in a function declaration. All assignment expressions defining
a function must be contiguous and separated by new lines, and the number of
patterns as arguments in each assignment expression must be the same.

As with definitions, only one declaration of a given function name can exist
in a given scope. However, the same function name can be declared multiple
times if each instance is in a different scope.

If a function declaration for some identifier $x$ occurs in scope $n$, then
a type signature for $x$ in scope $k>n$ is required. That is if a function has
been declared but its type has not been explicitly stated in the same or a higher
scope, a compile-time error will be generated. The type of the arguments
passed to a function are checked at compile-time as well, and an error
is issued if they don't match the types specified in that function's 
type signature.

\subsection{\texttt{main} Declaration}

Every SMURF program must define the reserved identifier \texttt{main}. This
identifier may only be used on the left-hand side of a top-level definition. The expression
bound to \texttt{main} is evaluated and its value is the value of the SMURF program itself.
That is, when a SMURF program is compiled and run, the expression bound to \texttt{main} is
evaluated and the result is converted to our bytecode representation of a MIDI file. As a result,
this expression must evaluate to a value of type \texttt{[]}, \texttt{Note}, \texttt{Chord}, 
\texttt{System}, or \texttt{[System]}. If a definition for \texttt{main} is not included in a 
SMURF program or if the expression bound to it does not have one of the types outlined above,
a compile-time error will occur.


\section{Library Functions}

Below are the library functions that can be used in the SMURF language. 
These functions are implemented in SMURF but are available to all users 
for their convenience. Each library function will include a description 
and its SMURF definition. \newline

\noindent\textbf{Head}

The function \texttt{head} takes a list and returns the first element. 
This function is commonly used when working with lists. The head function
is available for all lists through the use of polymorphic typing. 

\begin{verbatim}
head_note :: [a] -> a
head (h:tl) = h
\end{verbatim} 


\noindent\textbf{Tail}

The function \texttt{tail} takes a list and returns the end of the list.
This function is commmonly used when working with lists. 

\begin{verbatim}
tail_note :: [a] -> [a]
tail (h:tl) = tl
\end{verbatim} 


\noindent\textbf{MakeNotes}

The function \texttt{makeNotes} takes in three lists and returns a list 
of notes. The first list consists of expressions of type \texttt{Int} representing 
pitches or rests. The second list consists of expressions of type \texttt{Int}
representing the register that the pitch will be played in. The third list is a list 
of expressions of type \texttt{Beat} representing a set of durations. It is common in 
12 tone serialism to use lists of notes. This function allows the user 
to easily turn their modified rows and columns into a list of notes to add 
to their system. 

\begin{verbatim}
makeNotes :: [Int] -> [Int] -> [Beat] -> [Note]
makeNotes [] = []
makeNotes (h1:tl1) (h2:tl2) (h3:tl3) = (h1,h2)$h3:(makeNotes tl1 tl2 tl3)
\end{verbatim}

\clearpage

% citations begin here
\bibliographystyle{ieeetr}
\bibliography{ref/refs}


\end{document}
