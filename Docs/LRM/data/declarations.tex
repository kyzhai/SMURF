\section{Declarations and Bindings}
\setlength{\grammarindent}{4em}

This section of the LRM describes the syntax and informal semantics of
declarations in SMURF. Declarations may occur at a global level of scope
in a SMURF program or inside of
\texttt{let} expressions. The scoping of such declarations is described 
in this section. There are three types of declarations in SMURF: 
type signatures, definitions, and function declarations.

\subsection{Type Signatures}

\begin{grammar}

<type-sig> $\rightarrow$ <identifier> \texttt{::} <type>

<type> $\rightarrow$ \texttt{Int} \\ \texttt{Bool} \\ \texttt{Beat} \\ \texttt{Note} \\
											\texttt{Chord} \\ \texttt{System} \\ <identifier> \\ 
											\texttt{[} <type> \texttt{]} \\ \texttt{(} <type>\texttt{,} 
											$\ldots$\texttt{,} <type> \texttt{)} \\ <type> \texttt{->} <type>
											 \\ \texttt{(} <type> \texttt{)}
										

\end{grammar}

A type signature explicitly defines the type for a given identifier. The
\texttt{::} operator can be read as ``has type of." Only one type signature
for a given identifier can exist in a given scope. That is, two different
type signatures for a given identifier can exist, but they must be declared
in different scopes. There are three categories of types in SMURF: basic,  structured, and derived types; types are described in sections~\ref{sec:basictype}-\ref{sec:derivedtype}.

% type synonyms? 

SMURF also permits the use of polymorphism in specifying the types of identifiers. If some
non-keyword identifier appears on the right-hand side of the \texttt{::} operator, it means
that that identifier could be replaced with any standard SMURF type. For example, if we have
the type signature \texttt{x :: a -> b}, we have defined a function \texttt{x} that takes
an argument of any type and returns an argument of any type. That is, \texttt{x} could take
an argument of type \texttt{Int} and return a value of type \texttt{Char}.
If the same identifier is used multiple times in the same type signature, 
it must take on the same type everywhere that it appears. If our last example
was instead defined \texttt{x :: a -> a}, we would have defined a function that took an argument
of any type and returned a value of the \emph{same} type. That is, \texttt{x} could take an argument
of type \texttt{Int}, but then it would be forced to return a value of type \texttt{Int} as well.

%\subsubsection{Primitive Types}
%
%The four types \texttt{Int}, \texttt{Bool}, \texttt{Beat}, and \texttt{Note} are
%the fundamental building blocks of the type system in SMURF. \texttt{Int} and \texttt{Bool}
%are the traditional integer and boolean types. The \texttt{Beat} type is a subset of 
%\texttt{Int}, in that a value of type \texttt{Int} can be specified as type \texttt{Beat} if
%it is a power of two between 1 and 16. 
%The note type is written as \texttt{(Int, Int)\string^Beat}. The first expression of type
%\texttt{Int} is a pitch class, the next is a register, and the expression of type \texttt{Beat}
%is the durational value of the note.
%
%\subsubsection{Structured Types}
%
%SMURF has four structured types: lists, tuples, functions, and notes. Each
%type is represented by a special syntactic construct that operates on
%other types to generate a concrete structured result.
%
%The list type is written as \texttt{[} $t$ \texttt{]} which specifies the type of lists
%containing elements of type $t$.
%
%The tuple type is written as \texttt{($t_1$, $t_2$, \ldots, $t_n$)} where $t_i$
%can be any type. This specifies the type of tuples of size $n$ whose first
%element has type $t_1$, second element has type $t_2$, and so on. A tuple
%type must have at least two elements.
%
%The function type is written as $t_1$ \texttt{->} $t_2$ and specifies the type
%of functions that take an argument of type $t_1$ and return a value of type
%$t_2$. As with tuple types, $t_1$ and $t_2$ do not have to be the same.
%The function arrow is right-associative, so \texttt{Int -> Bool -> Bool} is
%equivalent semantically to \texttt{Int -> (Bool -> Int)}
%
%
%\subsubsection{Type Synonyms}
%
%Type synonyms give different names to specific types, making our language
%more readable and less verbose.
%The \texttt{Chord} type is equivalent to the \texttt{[Note]} type.
%The \texttt{System} type is equivalent to the \texttt{[Chord]} type.

\subsection{Definitions}

\begin{grammar}

<definition> $\rightarrow$ <identifier> $=$ <expression>

\end{grammar}

A definition binds an identifier to an expression. All definitions at a given
scope must be unique and can be mutually recursive. For example, the following
is legal in SMURF:

\begin{verbatim}
let x = 4
  	z = if y == 7 then x else y
  	y = let x = 5 
        in x + 3 
in x + z + y
\end{verbatim}


The $x$ in the nested let statement is in a different scope than the $x$ in the global let statement, 
so the two definitions do not conflict. $z$ is able to refer to $y$ even though
$y$ is defined after $z$ in the program. In this example, the three 
identifiers $x, y, $ and $z$ in the global \texttt{let} will evaluate to values 4, 8, and 8, respectively,
while the identifier $x$ after the nested let statement will evaluate to 5.

A type signature may be given for a definition but is not required.

\subsection{Function Declarations}

\begin{grammar}

<fun-dec> $\rightarrow$ <identifier> <args> $=$ <expression> \\
												 <func-dec> <new-line> <func-dec>

<args> $\rightarrow$ <pattern> \\ <pattern> <args>
\end{grammar}

A function declaration defines an identifier as a function that takes
some number of patterns as arguments and, based on which patterns are matched
when the function is called, returns the result
of a given expression. Essentially, a function declaration can be seen as a 
sequence of one or more assignment expressions, where each assignment defines
a value bound to the function identifier. The actual value returned by the function
is dependent on the patterns matched in the function call. There must be at least one pattern listed
as an argument in a function declaration. All assignment expressions defining
a function must be contiguous and separated by new lines, and the number of
patterns as arguments in each assignment expression must be the same.

As with definitions, only one declaration of a given function name can exist
in a given scope. However, the same function name can be declared multiple
times if each instance is in a different scope.

If a function declaration for some identifier $x$ occurs in scope $n$, then
a type signature for $x$ in scope $k>n$ is required. That is if a function has
been declared but its type has not been explicitly stated in the same or a higher
scope, a compile-time error will be generated. The type of the arguments
passed to a function are checked at compile-time as well, and an error
is issued if they don't match the types specified in that function's 
type signature.

\subsection{\texttt{main} Declaration}

Every SMURF program must define the reserved identifier \texttt{main}. This
identifier may only be used on the left-hand side of a top-level definition. The expression
bound to \texttt{main} is evaluated and its value is the value of the SMURF program itself.
That is, when a SMURF program is compiled and run, the expression bound to \texttt{main} is
evaluated and the result is converted to our bytecode representation of a MIDI file. As a result,
this expression must evaluate to a value of type \texttt{[]}, \texttt{Note}, \texttt{Chord}, 
\texttt{System}, or \texttt{[System]}. If a definition for \texttt{main} is not included in a 
SMURF program or if the expression bound to it does not have one of the types outlined above,
a compile-time error will occur.

