\section{Meaning of Identifiers}
In SMURF, an identifier is a name for a variable or function. The naming rule of
identifier is defined in section~\ref{sec:identifiers}. Keywords defined in
section~\ref{sec:keywords} are not available for the use of identifiers.


\subsection{Purpose}
\subsubsection{Functions}
Functions in SMURF enable users to structure programs in a more modular way. 
Each function has input and output, whose types need to be explicitly defined by
users. The function describles how to produce the output from its input.
SMURF is a side effect free language, which means
that if provided with the same input, a function is guaranteed to give the same
output. 


\subsubsection{variables}
In SMURF, a variable is an identifier that is linked to a value stored in the
system's memory or an expression that can be evaluated.
A variable is an abstraction of a computer memory cell or a collection
of memory cells. 
SMURF is a strongly typed programming language, which means the type of a variable can't
be changed once declared. Each variable has a static type which can be automatically
deduced by the SMURF compiler, or explicitly defined by users. The variables in SMURF are immutable.


\subsection{Scope and Lifetime}
In SMURF, a variable is bound in its scope to a value using constructs like
\texttt{let}. A variable is visible within its scope.
There is no global variables in SMURF. A variable becomes invalid after the 
ending of its scope. E.g.

\begin{lstlisting}
let prime = [2,0,4,6,8,10,1,3,5,7,9,11] 
in let prime = [0,2,4,6,8,10,1,3,5,7,9,11]
       p3 = trans 3 prime 
   in print (head p3)
\end{lstlisting}

where {\texttt prime} and {\texttt p3} are bound by the {\texttt let}
expression. 
In line 2, {\texttt prime} is
redefined in a inner {\texttt let} expression. In line 3, the {\texttt
trans} function sees the newly defined {\texttt prime}. So the result
to be printed in line 4 should be 3 (0+3). After the finishing of line
4, all the variables defined in this example will be invalid and can't be
accessed anymore.

\subsection{Basic Types}
There are three fundamental types in SMURF: {\texttt int}, {\texttt bool} and {\texttt bool}. 
\begin{itemize}
\item {\it Integer}: \texttt{Int}, used to represent integers.
\item {\it Boolean}: \texttt{Bool}, used to represent boolean values.
\item {\it Beat}   : \texttt{Beat}, used to represent the duration of a note. The value of {\texttt Beat} are integers of the power of 2, ranging from 1 to 16.
\end{itemize}

\subsection{Structured Types}
Structured types hold groups of elements. There are two structured types in
SMURF: {\it list} and {\it function}.

{\it list} has the format of 
\begin{grammar}
<list> $\rightarrow$ [{\it a, ..., n}]
\end{grammar}
where items {\it a - n} are elements in the list. Elements
of lists must have the same type.

{\it function} has the format of 
\begin{grammar}
<function> $\rightarrow$ {\it arg1} {\texttt -\textgreater} ... {\texttt
    -\textgreater} {\it argk}
{\texttt -\textgreater} {\it return-value}
\end{grammar}
where {\it arg1 - argk} are the input of function and {\it return-value} is the
output of the function.


\subsection{Derived Types}
Besides the basic types, SMURF has also several derived types. 

The type of \texttt{Note} is used to represent a musical note in SMURF. The format of \texttt{Note} is defined as below.
\begin{grammar}
<note> $\rightarrow$ ({\it pc}, {\it reg})\^{\it k}[.]* 
\end{grammar}
where {\it pc} is an integer in the range from -1 to 11. When {\it pc} 
has a value of -1, the note is a rest, otherwise it represents the pitch class of 
the note. 
{\it reg} is an integer in the range of 0-3, representing the register of the 
note, where the integer values and corresponding registers are given below.
\begin{itemize}
    \item 0: Bass clef, B directly below middle C to first C below middle C
    \item 1: Bass clef, next lowest B to next lowest C
    \item 2: Treble clef, middle C to the first B above middle C
    \item 3: Treble clef, first C above middle C to next highest B
\end{itemize}
{\it k} stands for the beat of the note.
Periods following {\it k} are optional. Users can add dots until the added duration
gets down to 16th note.

The type of \texttt{Chord} is used to represent several notes to be played simultaneously. It is defined as 
\begin{grammar}
<chord> $\rightarrow$ [{\it note}]
\end{grammar}
The compiler will check to make sure all the {\it note} in the same {\it chord} have the same time duration.

The type of \texttt{System} is used to represent a list of {\it chord} to be played in a sequential patterns. It is defined as 
\begin{grammar}
<system> $\rightarrow$ [{\it chord}]
\end{grammar}


\begin{comment}

\subsubsection{Pitch}
pc (pitch classes) are represented by integers ranging from 0 to 11.
\begin{itemize}
  \item A Note with pc = -1 represents a rest. In this special case, the register for the Note only matters in relation
  to whether the rest lies on the treble or bass clef (i.e. whether the register is positive or negative)
\end{itemize}

\subsubsection{Beat}
A Beat represents a length of musical time. It has a Time tag and integer type. 
\begin{itemize}
  \item Must have the string ``Time" followed by an integer that is a power of 2 and \textless\space 32 in declaration
  \begin{itemize}
    \item whole note: Time 1
    \item half note: Time 2
    \item quarter note: Time 4
    \item eighth note: Time 8
    \item sixteenth note: Time 16
    \item thirty-second note: Time 32
  \end{itemize}
  \item Uses + operator to combine Time but only adds two operands that contain the same integer; recursively 
  checks for Time operands that contain the same integers until only unequal Time integers are left
  \begin{itemize}
    \item Time 4 + Time 16 + Time 16 + Time 16 + Time 16
    \item Time 4 + Time 8 + Time 16 + Time 16
    \item Time 4 + Time 8 + Time 8 
    \item Time 4 + Time 4 = Time 2 (quarter note + quarter note = half note)
  \end{itemize}
\end{itemize}

\subsubsection{Register}
Registers are represented by integers ranging from 0 to 3.
\begin{itemize}
  \item \begin{music}  \trebleclef  \end{music}  Treble Clef: notes middle C and
  higher represented by 2 and 3  
    \begin{itemize}
    \item middle C to the first B above middle C: 2 
    \item first C above middle C to next highest B: 3
    \end{itemize}
  \item \begin{music} \bassclef  \end{music}  Bass Clef: notes lower than middle
  C represented by 0 and 1 
    \begin{itemize}
    \item B directly below middle C to first C below middle C: 0
    \item next lowest B to next lowest C: 1
    \end{itemize}
\end{itemize}

\subsubsection{Note}
A Note is a tuple of three integers and is declared as 
\begin{verbatim}
(pc: int, beat: Beat, register: int)
\end{verbatim}

\subsubsection{Chord}
A Chord is a list of notes and is declared as [Note]. The compiler will check that all notes in the list have 
the same beat count.
\end{comment}

% \subsubsection{Measure} 
% Measure are abandoned in lrm


