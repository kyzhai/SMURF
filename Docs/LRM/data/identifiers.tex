\section{Meaning of Identifiers}
In SMURF, an identifier is either a keyword or a name for a variable or a function. 
The naming rules for identifiers are defined in section~\ref{sec:identifiers}. This section outlines
the use and possible types of non-keyword identifiers.


\subsection{Purpose}
\subsubsection{Functions}
Functions in SMURF enable users to structure programs in a more modular way. 
Each function has at least one argument and exactly one return value, whose types need to be 
explicitly defined by the programmer. The function describes how to produce the return value,
given a certain set of arguments.
SMURF is a side effect free language, which means
that if provided with the same arguments, a function is guaranteed to return the same value. 


\subsubsection{Variables}
In SMURF, a variable is an identifier that is bound to a constant value
or to an expression. Any use of a variable within the scope of its definition
refers to the value or expression to which the variable was bound.
Each variable has a static type which can be automatically
deduced by the SMURF compiler, or explicitly defined by users. The variables in SMURF are immutable.


\subsection{Scope and Lifetime}
The lexical scope of a top-level binding in a SMURF program is the whole program itself.
As a result of this fact, a top-level binding can refer to any other top-level variable or function
on its right-hand side, regardless of which bindings occur first in the program.
Local bindings may also occur
with the \texttt{let} \emph{declarations} \texttt{in} \emph{expression} construct, and the
scope of a binding in \emph{declarations} is \emph{expression} and the right hand side of any other bindings in
\emph{declarations}. A variable or function is only visible within its scope.
An identifier becomes invalid after the ending of its scope. E.g.

\begin{lstlisting}
prime = [2,0,4,6,8,10,1,3,5,7,9,11] 
main = let prime = [0,2,4,6,8,10,1,3,5,7,9,11]
           p3 = (head prime) + 3
       in print (p3)
\end{lstlisting}

In line 1, \texttt{prime} is bound to a list of integers in a top-level definition, so it
has global scope.
In line 2, the \texttt{main} identifier (a special keyword described in \ref{sec:main}) 
is bound to a \texttt{let} expression. The \texttt{let} expression declares two local variables,
\texttt{prime} and \texttt{p3}. In line 3, the \texttt{head} function looks for a definition
of prime in the closest scope, and thus uses the binding in line 2. So the result
to be printed in line 4 should be 3. After line 4, the locally defined \texttt{prime} and
\texttt{p3} variables will be invalid and can't be accessed anymore.

\subsection{Basic Types}
\label{sec:basictype}
There are three fundamental types in SMURF: \texttt{Int}, \texttt{Bool} and \texttt{Beat}. 
\begin{itemize}
\item \texttt{Int}: integer type
\item \texttt{Bool}: Boolean type
\item \texttt{Beat}: beat type, used to represent the duration of a note. A constant of type
			\texttt{Beat} is any power of 2 ranging from 1 to 16.
\end{itemize}

\subsection{Structured Types}
\label{sec:structtype}
Structured types use special syntactic constructs and other types to describe new types. 
There are two structured types in SMURF: {\it list} types and {\it function} types.

A {\it list} type has the format \texttt{[}\emph{t}\texttt{]} where \emph{t} is a type that specifies the type
of all elements of the list. Thus, all elements of a list of type \texttt{[}\emph{t}\texttt{]}
must themselves have type \emph{t}. Note that \emph{t} itself may be a list type.

A {\it function} type has the format \emph{t$_1$} \texttt{->} \emph{t$_2$} \texttt {->}
\dots \texttt{->} \emph {t$_n$} \texttt{->} \emph{t$_{ret}$} which specifies a function type
that takes $n$ arguments, where the $kth$ argument has type \emph{t$_k$}, and returns an expression
of type \emph{t$_{ret}$}. Any type may be used to define a function type, except for a function
type itself. In other words, functions may not be passed as arguments to other functions, nor may
a function return another function.


\subsection{Derived Types}
\label{sec:derivedtype}
Besides the basic types, SMURF also has several derived types. 

Expressions of type \texttt{Note} are used to represent musical notes in SMURF.
The note type can be written as

\vspace{1em}
\texttt{(Int,Int)\$Beat}[\texttt{.}]* 
\vspace{1em}

The first expression of type \texttt{Int} must evaluate to an integer in the range from -1 to 11,
representing a pitch class or a rest. 
When this expression evaluates to -1, the note is a rest, otherwise it represents the pitch class of 
the note. The second expression of type \texttt{Int} must evaluate to an integer in the range of 0-3,
representing the register of the note, where the integer values and corresponding registers are given below.
\begin{itemize}
    \item 1: Bass clef, B directly below middle C to first C below middle C
    \item 0: Bass clef, next lowest B to next lowest C
    \item 2: Treble clef, middle C to the first B above middle C
    \item 3: Treble clef, first C above middle C to next highest B
\end{itemize}
The expression of type \texttt{Beat} refers to the duration of the note, and may be followed by
optional dots. The dot is a postfix operator described in section \ref{sec:postfixop}.
Using this format, a quarter note on middle C could be written as \texttt{(0,2)\$4}.

The \texttt{Chord} type is used to represent several notes to be played simultaneously.
It is equivalent to the list type \texttt{[Note]}.
The compiler will check to make sure all the notes in a chord have the same time duration.

The \texttt{System} type is used to represent a list of chords to be played sequentially.
It is equivalent to the list type \texttt{[Chord]}.


\begin{comment}

\subsubsection{Pitch}
pc (pitch classes) are represented by integers ranging from 0 to 11.
\begin{itemize}
  \item A Note with pc = -1 represents a rest. In this special case, the register for the Note only matters in relation
  to whether the rest lies on the treble or bass clef (i.e. whether the register is positive or negative)
\end{itemize}

\subsubsection{Beat}
A Beat represents a length of musical time. It has a Time tag and integer type. 
\begin{itemize}
  \item Must have the string ``Time" followed by an integer that is a power of 2 and \textless\space 32 in declaration
  \begin{itemize}
    \item whole note: Time 1
    \item half note: Time 2
    \item quarter note: Time 4
    \item eighth note: Time 8
    \item sixteenth note: Time 16
    \item thirty-second note: Time 32
  \end{itemize}
  \item Uses + operator to combine Time but only adds two operands that contain the same integer; recursively 
  checks for Time operands that contain the same integers until only unequal Time integers are left
  \begin{itemize}
    \item Time 4 + Time 16 + Time 16 + Time 16 + Time 16
    \item Time 4 + Time 8 + Time 16 + Time 16
    \item Time 4 + Time 8 + Time 8 
    \item Time 4 + Time 4 = Time 2 (quarter note + quarter note = half note)
  \end{itemize}
\end{itemize}

\subsubsection{Register}
Registers are represented by integers ranging from 0 to 3.
\begin{itemize}
  \item \begin{music}  \trebleclef  \end{music}  Treble Clef: notes middle C and
  higher represented by 2 and 3  
    \begin{itemize}
    \item middle C to the first B above middle C: 2 
    \item first C above middle C to next highest B: 3
    \end{itemize}
  \item \begin{music} \bassclef  \end{music}  Bass Clef: notes lower than middle
  C represented by 0 and 1 
    \begin{itemize}
    \item B directly below middle C to first C below middle C: 0
    \item next lowest B to next lowest C: 1
    \end{itemize}
\end{itemize}

\subsubsection{Note}
A Note is a tuple of three integers and is declared as 
\begin{verbatim}
(pc: int, beat: Beat, register: int)
\end{verbatim}

\subsubsection{Chord}
A Chord is a list of notes and is declared as [Note]. The compiler will check that all notes in the list have 
the same beat count.
\end{comment}

% \subsubsection{Measure} 
% Measure are abandoned in lrm


