\section{Library Functions}

\textcolor{red}{Since we moved the previous library functions to operations and removed measures not sure what to put here. I will brain storm and 
think of any library functions we might need.} 


Below are the library functions that can be used in the SMURF language. 
These functions are implemented in SMURF but are available to all users 
for their convenience. Each library function will include a description 
and its SMURF definition. \newline

\noindent\textbf{Head}

The function \texttt{head} takes a list and returns the first element. 
This function is commonly used when working with lists. 

\begin{verbatim}
head :: [a] -> a
head (h:tl) = h
\end{verbatim} 


\noindent\textbf{Tail}

The function \texttt{tail} takes a list and returns the end of the list.
This function is commmonly used when working with lists. 

\begin{verbatim}
tail :: [a] -> [a]
tail (h:tl) = tl
\end{verbatim}

\noindent\textbf{MakeNotes}

The function \texttt{makeNotes} takes in three lists and returns a list 
of notes. The first list corresponds to \texttt{integer}s representing 
pitches or rests. The second list corresponds to \texttt{integer}s representing the register that the pitch will be played in. The third list is a list 
of \texttt{Beat}s representing the duration of the note. It is common in 
12 tone serialism to use lists of notes. This function allows the user 
to easily turn their modified rows and columns into a list of notes to add 
to their system. 

\begin{verbatim}
makeNotes :: [a] [a] [b] -> [Note]
makeNotes [] = []
makeNotes (h1:tl1) (h2:tl2) (h3:tl3) = (h1,h2)^h3:(makeNotes tl1 tl2 tl3)
\end{verbatim}
