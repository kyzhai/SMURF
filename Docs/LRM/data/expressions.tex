\section{Expressions}

This section describes the syntax and semantics of \emph{expressions} in 
SMURF. Expressions can have either infix or prefix operators. All 
infix operators are built in operators. Unless otherwise stated infix 
operators are left associated. All prefix operators (function calls) 
are left associated. Prefix operators will always evaluate with a 
higher precedence than infix operators.  


%TODO: check the correct position for this table
\begin{center}
	\begin{tabular}{|c|c|}
		\hline
		Expression & Association \\
		\hline
		\texttt{f x + g y} & \texttt{(f x) + (g y)} \\
		\texttt{f g x} & \texttt{(f (g x))} \\
		\texttt{ let \{ ... \} in x + y} & \texttt{let \{ ... \} in (x + y)} \\
		type declaration example & order of precedence \\
		\hline
	\end{tabular}
\end{center}


\subsection{Errors}

Errors can occur during runtime of a SMURF program. These erros cannot be 
caught by a programmer nor can a programmer produce an error artificially. 
Errors that occur during runtime are out of bounds errors in \texttt{Note} 
and \texttt{Beat} creation.

\subsection{Description of Precedence}


\subsection{Primary Expressions}

\subsubsection{Identifiers}
An identifier expression is an expression that involves only identifiers. 
\begin{verbatim}
identifier-expression :: identifiers
\end{verbatim}
where \texttt{identifiers} are defined in section~\ref{sec:identifiers}.


\subsubsection{Constants}
A constant expression is an expression that involves only constants. 
\begin{verbatim}
constant-expression :: constants
\end{verbatim}
where \texttt{constants} are defined in section~\ref{sec:constants}.

\subsubsection{(Expression)}
An expression surrounded by parentheses is a new expression.
\begin{verbatim}
paren-expression :: ( expression )
\end{verbatim}


\subsection{Curried Applications}

    \subsubsection{Function declaration}
    \emph{function-expression} \texttt{::}  \emph{argument-type-list} \texttt{$\rightarrow$} \emph{result-type} \\ \\
    where\\
    
    \emph{argument-type-list} $\rightarrow$ \emph{argument-type}$+$ \\ \\
    A function declaration must be on its own line and must declare a type. Declaring a general type is allowed. There are no explicit return statements.
  
    \subsubsection{Function application}
    \emph{function-expression} \emph{argument-expression-list} \\ \\
    where\\
    
    \emph{argument-expression-list} $\rightarrow$ \emph{argument-expression}$+$ \\ \\ 
    A function application associates from left to right, so parentheses are optional: \\
    
    \emph{funct a b}  is equivalent to \emph{((funct a) b)} \\ \\
    Parentheses are used to change the precedence from the default. The following evaluates function \emph{funct1} with argument \emph{b} then 
    evaluates function \emph{funct2} with argument \emph{a}: \\
    
    \emph{funct2 a (funct1 b)}

  \subsubsection{**PARTIAL APPLICATION**}
  

\subsection{Operator Application}
    \emph{expression  operator  expression}\\ \\
    where\\
    
    \emph{operator} $\rightarrow$ \emph{arithmetic-operator} $ | $  \emph{comparison-operator} $ | $ \emph{boolean-operator} $ | $ 
    
    \emph{list-operator} $ | $ \emph{function-operator}
    
\subsection{Conditionals}
  \texttt{if}  \emph{expression}  \texttt{then}  \emph{expression-true}  \texttt{else}  \emph{expression-false} \\ \\
  When the value of expression evaluates to true,  \emph{expression-true} is evaluated, otherwise  \emph{expression-false} is evaluated. Conditional expressions do not have newline restrictions.

\subsection{Lists}
  \texttt{[} \emph{expression-list} \texttt{]} \\ \\
  where \\
  
  \emph{expression-list} $\rightarrow$  \emph{expression}$?$ (\texttt{,} \emph{expression})$*$ \\ \\
\texttt{[}\emph{expression$_{1}$}\texttt{,} \emph{...}\texttt{,} \emph{expression$_{k}$}\texttt{]} = 
  \emph{expression$_{1}$}\texttt{:(}\emph{expression$_{2}$}\texttt{:(} \emph{...} \texttt{(}\emph{expression$_{k}$}\texttt{:[ ]))} \\ 
where \textit{k} \textgreater\space 0. The expressions in a list must all be of the same type. Both the list constructor \texttt{:} and empty list \texttt{[ ]} are reserved as part of the language syntax and therefore cannot be hidden or redefined. The list constructor has right associativity.

\subsection{Tuples}
  \texttt{(} \emph{expression-list} \texttt{)} \\ \\
  where \\
  
  \emph{expression-list} $\rightarrow$  \emph{expression} (\texttt{,} \emph{expression})$*$  \texttt{,} \emph{expression} \\ \\
The expressions in a tuple may be of different types. The constructor of an n-tuple is denoted by (\textunderscore
, ..., \textunderscore) where there are \textit{n-1} commas.

\subsection{Parenthesized Expressions}
  \texttt{(} \emph{expression} \texttt{)} \\ \\
  where expression is evaluated as a primary expression.

\subsection{Expression Type-Signature}
  \emph{expression} \texttt{::} \emph{type} \\ \\
  where \emph{expression} is an expression and \emph{type} is a type. This is used to explicitly define a type for an expression. The declared type may be more specific than the principal type but it is illegal to give a declared type that is more general than the principal type. Giving a declared type that is not comparable to the principal type will also yield an error.

\subsection{Let Expressions}
\begin{grammar}
<let-exp> $\rightarrow$ let <decls> in <expression>
\end{grammar}
Let expressions have the form \emph{let \{} $d_1;...;d_n$ \} \emph{in e}, where $d_n$ is the nth declaration and \emph{e} is an expression. Note that the scope of $d_n$ is in \emph{e}.

% are pattern bindings matched lazily?

\subsection{Pattern Matching}

Patterns can be found in function definitions, pattern bindings, and list operations. Patterns are matched with values. Matching a pattern can either be successful or it can fail. A successful match will return  a binding for the variables in the pattern. 

\subsection{Guards}
\begin{grammar}
<guard> $\rightarrow$  let <decls> | <infixexp>          
\end{grammar}

There are two kinds of guards in SMURF: local bindings and boolean guards. Local bindings have the form let $<decls>$ and introduces the declarations to the program environment. Boolean guards are expressions of type Bool. The boolean guard succeeds if the expression evaluates to True.

