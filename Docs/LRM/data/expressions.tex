\section{Expressions}

This section describes the syntax and semantics of \emph{expressions} in 
SMURF. Expressions in SMURF use prefix, infix, or postfix operators.
Unless otherwise stated, all infix and postfix operators are left-associative and
all prefix operators are right-associative. Some examples of association are given below.


%TODO: check the correct position for this table
\begin{center}
	\begin{tabular}{|c|c|}
		\hline
		Expression & Association \\
		\hline
		\texttt{f x + g y - h z} & \texttt{((f x) + (g y)) - (h z)} \\
		\texttt{f g x} & \texttt{((f g) x)} \\
		\texttt{ let \{ ... \} in x + y} & \texttt{let \{ ... \} in (x + y)} \\
		\texttt{$\sim$ <> [0,1,2,3,4,5,6,7,8,9,10,11]} &
					\texttt{($\sim$ (<> [0,1,2,3,4,5,6,7,8,9,10,11]))}\\
		\hline
	\end{tabular}
\end{center}

%
%\subsection{Errors}
%
%Errors can only occur during compilation of a SMURF program. All generated 
%code will run successfully. 
%

\subsection{Description of Precedence in Expressions}
Precedence describes the order of evaluation for subexpressions in a given expression.
The order of precedence for expressions and their operators mirrors our ordering of the following
subsections i.e. the first subsection describes the expressions with highest precedence, and the last
subsection describes the expressions with lowest precedence.

\subsection{Primary Expressions}

\setlength{\grammarindent}{7.5em}
\begin{grammar}
<primary-expr> $\rightarrow$ <variable> \\ <constant> \\ \texttt{(} <expression> \texttt{)} \\
														\texttt{[} (<expression>?|<expression> (\texttt{,} <expression>)*) \texttt{]} \\
														\texttt{(} <expression>\texttt{,} <expression>\texttt{)\$}<expression>

<variable> $\rightarrow$ <identifier> 

\end{grammar}

\subsubsection{Variables}
A variable $x$ is a primary expression whose type is the same as the type of $x$.
When we evaluate a variable, we are actually evaluating
the expression bound to the variable in the closest lexical scope.

\subsubsection{Constants}
An integer or boolean constant, as described in section \ref{sec:constants}, is a primary expression
with type equivalent to the type of the constant.

\subsubsection{Parenthesized Expression}
An expression surrounded by parentheses is a primary expression.

\subsubsection{Lists}
A list is a primary expression. Lists can be written as: \\

\texttt{[}\emph{expression$_{1}$}\texttt{,} \emph{...}\texttt{,} \emph{expression$_{k}$}\texttt{]} \\

	or \\

  \emph{expression$_{1}$}\texttt{:(}\emph{expression$_{2}$}\texttt{:(} \emph{...} \texttt{(}\emph{expression$_{k}$}\texttt{:[]))} \\ \\
where \textit{k} $>=$ 0. These two lists are equivalent.
The expressions in a list must all be of the same type. The empty list \texttt{[]} is typeless until it is applied to a non-empty
list of some type \texttt{[}\emph{t}\texttt{]}, at which time it is assigned the same type.


\subsubsection{Notes}
A note is a primary expression, and is written as a tuple of expressions of type \texttt{Int} followed
by a \texttt{\$} symbol and an expression of type \texttt{Beat}. The values of each of these
expressions must follow the rules outlined in section \ref{sec:derivedtype}.

\subsection{Postfix Operator Expressions}
\label{sec:postfixop}

\begin{grammar}
<postfix-expression> $\rightarrow$ <beat-expr>\texttt{.}
\end{grammar}

The only expression in SMURF using a postfix operator is the partial augmentation of an expression
of type \texttt{Beat}, which uses the dot operator.
We say ``partial augmentation" because a dot increases the durational value of
the expression to which it is applied, but only by half of the durational value of that expression.
That is, if \emph{expr} is an expression of type \texttt{Beat} that evaluates to a duration of $n$, 
then \emph{expr}\texttt{.} is a postfix expression of type \texttt{Beat} 
that evaluates to a duration of $n/2$.
The dot operator may be applied until it represents an addition of a sixteenth note
duration, after which no more dots may be applied. For instance, \texttt{4..} is legal, as this
is equivalent to a quarter note duration plus an eighth note duration (the first dot) plus a
sixteenth note duration (the second dot). However, \texttt{8..} is not legal, as the second
dot implies that a thirty-second note duration should be added to the total duration of this
expression. Our compiler will check the number of dots and return an error if too many are applied.

\subsection{Prefix Operator Expressions}
\label{sec:prefixop}
\begin{grammar}
<prefix-expression> $\rightarrow$ <prefix-op> <expression>
\end{grammar}


\begin{table} [H]
\centering
\begin{tabularx}{0.9\textwidth}{llX}
\hline\hline
Prefix Operator & Description & Example \\
\hline\hline
   \texttt{$\sim$} & Tone row inversion & $\sim$ \texttt{row} (returns the inversion of \texttt{row})\\ \hline
   \texttt{<>} & Tone row retrograde & \texttt{<>}~\texttt{row} (returns the
           retrograde of \texttt{row})\\ \hline
   \texttt{!} & Logical negation & \texttt{if !(a == 5) then True else False} \\ \hline
\end{tabularx}
\end{table}

SMURF has three prefix operators: logical negation, tone row inversion, and tone row retrograde.
There is another row transformation operator, but it takes multiple arguments and is described in
section \ref{sec:bintonerow}. The logical negation operator can only be applied to expressions of
type \texttt{Bool}, and the two row operators can only be applied to expressions of type
\texttt{[Int]}. The compiler will check that all of the integers in a list are in the range $0-11$
if the list is passed to either of the tone row operators.
All three operators return an expression of the same type as the expression the
operator was applied to.

\subsection{Binary Operator Expressions}
\label{sec:binaryop}
\begin{grammar}
<binary-expression> $\rightarrow$ <expression>$_1$ <binary-op> <expression>$_2$
\end{grammar}
The following categories of binary operators exist in SMURF, and are listed in order of decreasing
precedence: list, arithmetic, comparison, boolean, tone row. 

\subsubsection{List operators}

\begin{table} [H]
\centering
\begin{tabular}{lll}
\hline\hline
List Operator & Description & Example \\
\hline\hline
   \texttt{++} & List Concatenation & \texttt{[1,2,3] ++ [4,5,6]} (result is \texttt{[1,2,3,4,5,6]}) \\ \hline
   \texttt{:} & List Construction & \texttt{1 : [2,3,4]} (result is \texttt{[1,2,3,4]}) \\ \hline
\end{tabular}
\end{table}

List operators are used to construct and concatenate lists. 
These two operators are \texttt{:} and \texttt{++}, respectively. The \texttt{:} operator has 
higher precedence than the \texttt{++} operator. Both of these operators are right-associative.
List operators require that  \emph{expression$_{2}$} be an expression of type 
\texttt{[}\emph{t}\texttt{]}, where \emph{t} is the type of \emph{expression$_{1}$}. 
In other words, \emph{expression$_{1}$} must have the same type 
as the other elements in \emph{expression$_{2}$} when doing list construction.
When doing list concatenation, both
\emph{expression$_{1}$} and \emph{expression$_{2}$} must have type \texttt{[}\emph{t}\texttt{]},
where \emph{t} is some non-function type. 

\subsubsection{Arithmetic operators}

\begin{table} [H]
\centering
\begin{tabularx}{\textwidth}{lXX}
\hline\hline
Arithmetic Operator & Description & Example \\
\hline\hline
  \texttt{+} & Integer Addition  & \texttt{a + 2} \\ \hline
  \texttt{-} & Integer Subtraction  & \texttt{5 - a} \\ \hline 
  \texttt{*} & Integer Multiplication  & \texttt{5 * 10} \\ \hline 
  \texttt{/} & Integer Division  & \texttt{4 / 2} \\ \hline 
  \texttt{\%} & Integer Modulus, ignores negatives  & \texttt{14 \% 12} \\ \hline
  \texttt{\%+} & Pitch Class Addition (addition mod 12)  & \texttt{14 \%+ 2 == 4 }\\ \hline
  \texttt{\%-} & Pitch Class Subtraction (subtraction mod 12)  & \texttt{14 \%- 2 == 0 } \\ \hline
  \texttt{$\$$+} & Rhythmic Addition & \texttt{2 $\$$+ 2 == 1} \\ \hline
  \texttt{$\$$-} & Rhythmic Subtraction & \texttt{1 $\$$- 2 == 2}  \\ \hline
  \texttt{$\$$*} & Rhythmic Augmentation & \texttt{8 $\$$* 4 == 2}  \\ \hline
  \texttt{$\$$/} & Rhythmic Diminution & \texttt{2 $\$$/ 8 == 16}  \\ \hline
\end{tabularx}
\end{table}

There are three types of arithmetic operators (listed in descending order of precedence):
basic, pitch class, and rhythmic. 
Basic arithmetic operators are those found in most programming languages like 
\texttt{+}, \texttt{-}, \texttt{*}, \texttt{/}, and \texttt{\%}, which operate on expressions of
type \texttt{Int}. It should be noted that the modulus operator ignores negatives e.g.
\texttt{13 \% 12} is equal to \texttt{-13 \% 12} is equal to \texttt{1}.
The pitch class operators are \texttt{\%+} and \texttt{\%-}. These can be read as mod 12 addition
and mod 12 subtraction. They also operate on expressions of type \texttt{Int}, but the expressions
must evaluate to values in the range $0-11$. The built-in mod 12
arithmetic serves for easy manipulation of pitch class integers.
Lastly, there are rhythmic arithmetic operators (both operands must be of type \texttt{Beat}). 
These include \texttt{\$+}, \texttt{\$-}, \texttt{\$*}, and \texttt{\$/}. 

Out of the basic arithmetic operators, \texttt{*}, \texttt{/}, and \texttt{\%} 
have higher precedence than \texttt{+} and \texttt{-}. 
Both pitch class operators have the same precedence. 
Rhythmic arithmetic operators follow the same form as basic arithmetic operators, with
\texttt{\$*} and \texttt{\$/} having higher precedence than \texttt{\$+} and {\$-}.

\subsubsection{Comparison operators}

\begin{table} [H]
\centering
\begin{tabularx}{\textwidth}{XlX}
\hline\hline
Comparison Operator & Description & Example \\
\hline\hline
  \texttt{<}  & Integer Less than & \texttt{if a < 5 then True else False} \\ \hline
  \texttt{>}  & Integer Greater than & \texttt{if a > 5 then True else False}  \\ \hline
  \texttt{<=}  & Integer Less than or equal to & \texttt{if a <= 5 then True else False} \\ \hline
  \texttt{>=} & Integer Greater than or equal to & \texttt{if a >= 5 then True else False} \\ \hline
  \texttt{\$<} & Rhythmic Less than & \texttt{4 \$< 8 == False} \\ \hline
  \texttt{\$>}  & Rhythmic Greater than &  \texttt{4 \$> 8 == True}  \\ \hline
  \texttt{\$<=} & Rhythmic Less than or equal to & \texttt{4 \$<= 4 == True} \\ \hline
  \texttt{\$>=} & Rhythmic Greater than or equal to &  \texttt{1 \$>= 16 == True} \\ \hline
  \texttt{==} & Structural comparison & \texttt{if a == 5 then a = True else a = False} \\ \hline
\end{tabularx}
\end{table}

SMURF allows comparison operations between expressions of type \texttt{Int} or \texttt{Beat}.
Structural comparison, however, can be used to compare expressions of any type for equality.
All of the comparison operators have the same precedence except for structural comparison, which
has lower precedence than all of the other comparison operators.

\subsubsection{Boolean operators}


\begin{table} [H]
\centering
\begin{tabular}{lll}
\hline\hline
Boolean Operator & Description & Example \\
\hline\hline
   \texttt{\&\&} & Logical conjunction & \texttt{if b \&\& c  then True else False} \\ \hline
   \texttt{\textbar\textbar} & Logical disjunction & \texttt{if b \textbar\textbar\space   c  then True else False} \\ \hline
 \end{tabular}
\end{table}

Boolean operators are used to do boolean logic on expressions of type \texttt{Bool}. Logical
conjunction has higher precedence than logical disjunction.

\subsubsection{Tone row operators}
\label{sec:bintonerow}
The only binary tone row operator is the transposition operator, \texttt{$\wedge\wedge$}.
\emph{expression$_1$} must be an expression that evaluates to a list of pitch classes,
and \emph{expression$_2$} must have type \texttt{Int}. The result of this operation is a new
tone row where each pitch class has been transposed up by $n$ semitones, where $n$ is the result
of evaluating \emph{expression$_2$}.

\subsection{Conditional expressions}
\begin{grammar}
<conditional-expression> $\rightarrow$ \texttt{if} \emph{expression$_{boolean}$} \texttt{then}
													\emph{expression$_{true}$} \texttt{else} 
													\emph{expression$_{false}$}
\end{grammar}
When the value of \emph{expression$_{boolean}$} evaluates to true, \emph{expression$_{true}$} 
is evaluated, otherwise \emph{expression$_{false}$} is evaluated. \emph{expression$_{boolean}$}
must have type \texttt{Bool}. Conditional expressions do not have newline restrictions.

\subsection{Let Expressions}
\begin{grammar}
<let-exp> $\rightarrow$ \texttt{let} <decls>+ \texttt{in} <expression>
\end{grammar}
Let expressions have the form \texttt{let} \emph{decls} \texttt{in} \emph{e}, where 
\emph{decls} is a list of one or more declarations and \emph{e} is an expression. 
The scope of these declarations is discussed in section \ref{sec:declarations}.

The declarations in a let expression can be separated by an \& symbol and also by a newline character. For example:

\begin{verbatim}
let x = 2 & y = 4 & z = 8 
in x + y + z
\end{verbatim}

The previous code is equivalent to the following:

\begin{verbatim}
let x = 2
    y = 4
    z = 8
in x + y + z
\end{verbatim}

If the first code snippet were written without the \& symbol and no newlines in between the
declarations, a compile-time error would be raised.

\subsection{Function application expressions}

\begin{grammar}
<function-app-expression> $\rightarrow$ <identifier> <expression>+
\end{grammar}

A function gets called by invoking its name and supplying any necessary arguments. 
Functions can only be called if they have been declared in the same scope where the call occurs,
or in a higher scope. Functions may be called recursively. Function application associates 
from left to right. Parentheses can be used to change the precedence from the default.
The following evaluates function \emph{funct1} with argument \emph{b} then evaluates function
\emph{funct2} with argument \emph{a} and the result from evaluating (\emph{funct1 b}): \\
    
    \emph{funct2 a (funct1 b)}\\ \\
If the parentheses were not included, a compile-time error would be generated, as it would imply
that \emph{funct2} would be called with \emph{a} as
its first argument and \emph{funct1} as its second argument, which is illegal based on the description
of function types in section \ref{sec:structtype}.

A function call may be used in the right-hand side of a binding just like any other expression. 
For example:
\begin{verbatim}
let a = double 10 
in a
\end{verbatim}
evaluates to 20, where \texttt{double} is a function that takes a single integer argument and
returns that integer multiplied by two. 

%\subsection{Guards}
%\begin{grammar}
%<guard> $\rightarrow$  let <decls> | <infixexp>          
%\end{grammar}
%
%There are two kinds of guards in SMURF: local bindings and boolean guards. Local bindings have the form let $<decls>$ and introduces the declarations to the program environment. Boolean guards are expressions of type Bool. The boolean guard succeeds if the expression evaluates to True.
%
