\section{Lexical Conventions}
SMURF programs are lexically composed of three elements: comments, tokens, and whitespace.

\subsection{Comments}
SMURF allows nested, multiline comments in addition to single line comments.
\begin{table} [H]
\centering
\begin{tabularx}{\textwidth}{lXl}
\hline\hline
Comment Symbols & Description & Example \\
\hline\hline
  \texttt{/* */} & Multiline comments, nesting allowed & \texttt{/* This /* is all */ commented */} \\ \hline
  \texttt{//} & Single-line comment & \texttt{// This is a comment} \\ \hline
\end{tabularx}
\end{table}


\subsection{Tokens}
In SMURF, a token is a string of one or more characters that is significant as a group.
SMURF has 6 kinds of tokens: {\it identifiers}, {\it keywords}, {\it constants},
      {\it operators},
{\it separators} and {\it newlines}.

\subsubsection{Identifiers}
\label{sec:identifiers}
An identifier consists of a letter followed by other letters, 
digits and underscores. The letters are the ASCII characters \texttt{a}-\texttt{z} and
\texttt{A}-\texttt{Z}. Digits are ASCII characters \texttt{0}-\texttt{9}. SMURF is case sensitive.

\begin{grammar}
<letter> $\rightarrow$ [`a'-`z' `A'-`Z'] 

<digit> $\rightarrow$ [`0'-`9'] 

<underscore> $\rightarrow$ {`_'} 

<identifier> $\rightarrow$ <letter> (<letter> | <digit> | <underscore>)*
\end{grammar}

\subsubsection{Keywords}
\label{sec:keywords}
Keywords in SMURF are identifiers reserved by the language. Thus, they are not available for
re-definition or overloading by users. 

\begin{table} [H]
	\centering
    \begin{tabular}{ll}
    \hline\hline
    Keywords & Descriptions \\ 
    \hline\hline
      \texttt{Bool} & Boolean data type \\ \hline
      \texttt{Int} & Integer data type \\ \hline
      \texttt{Note} & Atomic musical data type \\ \hline
      \texttt{Beat} & Note duration data type\\ \hline
      \texttt{Chord} & Data type equivalent to \texttt{[Note]} type \\ \hline
      \texttt{System} & Data type equivalent to \texttt{[Chord]} type \\ \hline
      \texttt{True, False} & Boolean constants \\ \hline
      \texttt{let, in} & Allow local bindings in expressions  \\ \hline
      \texttt{if, then, else} & Specify conditional expression, else compulsory  \\ \hline
      \texttt{random} & Generate random numbers \\ \hline
      \texttt{print} & Print information to standard output \\ \hline
      \texttt{main} & Specify the value of a SMURF program\\ \hline
    \end{tabular}
\end{table}


\subsubsection{Constants}
\label{sec:constants}
In SMURF, constants are expressions with a fixed value. Integer literals and
Boolean keywords are the constants of SMURF. 

\setlength{\grammarindent}{6em}
\begin{grammar}
<digit> $\rightarrow$ [`0'-`9'] 

<constant> $\rightarrow$ \texttt{-}? [`1'-`9'] <digit>* \\
												 \texttt{0} <digit>* \\
												\texttt{True} \\
												\texttt{False}
\end{grammar}

\subsubsection{Operators}
SMURF permits arithmetic, comparison, boolean, list, declaration, and row operations, all of which
are carried out through the use of specific operators. The syntax and semantics of all of these
operators are described in sections \ref{sec:postfixop}, \ref{sec:prefixop}, and \ref{sec:binaryop},
except for declaration operators, which are described in section \ref{sec:declarations}.


\subsubsection{Newlines}
SMURF uses newlines to signify the end of a declaration, except
when preceded by the \texttt{\textbackslash} token. In the latter case, the newline is ignored by the compiler 
(see example below). If no such token precedes a newline, then the compiler will treat the newline as
a token being used to terminate a declaration.

\subsubsection{Separators}

\begin{grammar}
<separator> $\rightarrow$ \texttt{,} \\
												  \texttt{\&} \\
													\texttt{\textbackslash}
\end{grammar}

Separators in SMURF are special tokens used to separate other tokens. 
Commas are used to separate elements in a list.
The \texttt{\&} symbol can be used in place of a newline. That is, the compiler
will replace all \texttt{\&} characters with newlines. The
\texttt{\textbackslash} token, when followed by a newline token,
may be used to splice two lines. E.g.
\begin{lstlisting}
genAltChords (x:y:ys) = [(x,Time 4,1)]   \
                        :[(y,Time 4,-1)]:genAltChords ys
\end{lstlisting}
is the same as 
\begin{lstlisting}
genAltChords (x:y:ys) = [(x,Time 4,1)]:[(y,Time 4,-1)]:genAltChords ys
\end{lstlisting}


\subsection{Whitespace}
\label{sec:whitespaces}
{\it Whitespace} consists of any sequence of {\it blank} and {\it tab} characters.
Whitespace is used to
separate tokens and format programs. All whitespace is ignored by the
SMURF compiler. As a result, indentations are not significant in SMURF.

