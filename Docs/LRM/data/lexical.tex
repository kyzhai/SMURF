\section{Lexical Conventions}
\subsection{A High Level Description of SMURF Programs}
SMURF is a function language that allows a composer to create serialist music
based on twelve-tone composition. In general, serialism is a musical composition
technique where a set of values, chosen through some methodical progress,
          generates a sequence of musical elements. SMURF is based on the
          functional syntax and semantics set forth by Haskell. The backend of
          SMURF generates MIDIs corresponding to the uses's initial program in
          SMURF. 

\subsection{Tokens}
SMURF has 5 kinds of tokens: identifiers, keywords, constants, operators and separators.

\subsubsection{Identifiers}
\label{sec:identifiers}
An identifier consists of a letter followed by other letters and
digits. The letters are the ascii characters a-z, A-Z and \_. Digits are ascii
characters 0-9. SMURF is case sensitive.
\begin{verbatim}
letter -> [`a'-`z' `A'-`Z']
digit -> [`0'-`9']
identifiers -> letter (letter | digit | `_')*
\end{verbatim}

\subsubsection{Keywords}
This is a list of reserved keywords in SMURF. The keywords are used by the
language, thus are not avaliable for re-definition or overloading.
\begin{table} [H]
	\centering
    \begin{tabular}{ll}
    \hline\hline
    Keywords & \\ 
    \hline\hline
      let & Specify variables and functions  \\ \hline
      in & Allow local variable binding in expression \\ \hline
      if, then, else & Specify conditional expression, else compulsory  \\ \hline
      True, False & Specify boolean logic \\ \hline
      otherwise & Specify conditional expression used with guards \\ \hline 
      %genScore & Generate musical score given list of measures as argument  \\ \hline
    \end{tabular}
\end{table}


\subsubsection{Constants}
\label{sec:constants}
In SMURF, constants are expressions with a fixed value. Integer literals and
boolean keywords are the constants of SMURF. 

\begin{verbatim}
letter -> [`a'-`z' `A'-`Z']
digit -> [`0'-`9']
constants -> [`1'-`9'] digit* | True | False
\end{verbatim}


\subsubsection{Operators}
Operators in SMURF can be classified into comment operators, arithmetic operators, comparision
operators, boolean operators, list operators, declaration operators, and row operators. 

SMURF allows nested, multiline comments in addition to single line comments.
\begin{table} [H]
\centering
\begin{tabularx}{\textwidth}{lXl}
\hline\hline
Comment Operator & Description & Example \\
\hline\hline
  \texttt{/* */} & Multiline comments, nesting allowed & \texttt{/* This /* is all */ commented */} \\ \hline
  \texttt{//} & Single-line comment & \texttt{// This is a comment} \\ \hline
\end{tabularx}
\end{table}

SMURF allows addition, subtraction, and modulus on expressions that evaluate to integer literals of type \texttt{Int}. We also have addition, subtraction,
multiplication, and division on expressions that evaluate to integer literals of type \texttt{Beat}. All of these operators are infix.
The modulus operator ignores negatives e.g. \texttt{13 \% 12} is equal to \texttt{-13 \% 12} is equal to \texttt{1}. Rhythmic arithmetic is explained
in section INSERT SECTION NUMBER HERE.
\begin{table} [H]
\centering
\begin{tabularx}{\textwidth}{lXX}
\hline\hline
Arithmetic Operator & Description & Example \\
\hline\hline
  \texttt{+} & Integer Addition  & \texttt{a + 2} \\ \hline
  \texttt{-} & Integer Subtraction  & \texttt{5 - a} \\ \hline 
  \texttt{\%} & Integer Modulus, ignores negatives  & \texttt{14 \% 12} \\ \hline
  \texttt{$\wedge$+} & Rhythmic Addition & \texttt{2 $\wedge$+ 2 == 1} \\ \hline
  \texttt{$\wedge$-} & Rhythmic Subtraction & \texttt{1 $\wedge$- 2 == 2}  \\ \hline
  \texttt{$\wedge$*} & Rhythmic Augmentation & \texttt{8 $\wedge$* 4 == 2}  \\ \hline
  \texttt{$\wedge$/} & Rhythmic Diminution & \texttt{2 $\wedge$/ 8 == 16}  \\ \hline
\end{tabularx}
\end{table}

SMURF allows comparison operations between expressions that evaluate to values of type \texttt{Int} or \texttt{Beat}.
\begin{table} [H]
\centering
\begin{tabular}{lll}
\hline\hline
Comparison Operator & Description & Example \\
\hline\hline
  \texttt{<}  & Integer Less than & \texttt{if a \textless\space  5 then True else False} \\ \hline
  \texttt{>}  & Integer Greater than & \texttt{if a \textgreater\space  5 then True else False}  \\ \hline
  \texttt{<=}  & Integer Less than or equal to & \texttt{if a \textless= 5 then True else False} \\ \hline
  \texttt{>=} & Integer Greater than or equal to & \texttt{if a \textgreater= 5 then True else False} \\ \hline
  \texttt{\$<} & Rhythmic Less than & \texttt{4 \textless\space  8 == False} \\ \hline
  \texttt{\$>}  & Rhythmic Greater than &  \texttt{4 \textgreater\space  8 == True}  \\ \hline
  \texttt{\$<=} & Rhythmic Less than or equal to & \texttt{4 \textless= 4 == True} \\ \hline
  \texttt{\$>=} & Rhythmic Greater than or equal to &  \texttt{1 \textgreater= 16 == True} \\ \hline
\end{tabular}
\end{table}


SMURF allows logical negation, conjunction, and disjunction, in addition to structural comparison.
\begin{table} [H]
\centering
\begin{tabular}{lll}
\hline\hline
Boolean Operator & Description & Example \\
\hline\hline
   \texttt{==} & Structural comparison & \texttt{if a == 5 then a = True else a = False} \\ \hline
   \texttt{!} & Logical negation & \texttt{if !(a == 5) then True else False} \\ \hline
   \texttt{\&\&} & Logical conjunction & \texttt{if b \&\& c  then True else False} \\ \hline
   \texttt{\textbar\textbar} & Logical disjunction & \texttt{if b \textbar\textbar\space   c  then True else False} \\ \hline
 \end{tabular}
\end{table}

SMURF allows concatenation and construction of lists.
\begin{table} [H]
\centering
\begin{tabular}{lll}
\hline\hline
List Operator & Description & Example \\
\hline\hline
   \texttt{++} & List Concatenation & \texttt{[1,2,3] ++ [4,5,6]} (result is \texttt{[1,2,3,4,5,6]}) \\ \hline
   \texttt{:} & List Construction & \texttt{1 : [2,3,4]} (result is \texttt{[1,2,3,4]}) \\ \hline
\end{tabular}
\end{table}


SMURF provides operators to specify types of identifiers, arguments, and return values in various declarations,
as well as an operator to indicate a guard in a declaration.
\begin{table} [H]
\centering
\begin{tabularx}{\textwidth}{lXl}
\hline\hline
Declaration Operator & Description & Example \\
\hline\hline
   \texttt{::} & Type specification & \texttt{number :: Int} \\ \hline
   \texttt{->} & Argument and function return type specification
     & \texttt{isPositiveNum :: Int \textendash\textgreater\space Bool}  \\ \hline
   \texttt{\textbar} & Gaurd indicator & \texttt{isSeven :: [Int] \textendash\textgreater\space Bool}\\ 
	 	 && \texttt{isSeven num} \\
     && \texttt{\textbar\space (num == 7) = True} \\
     && \texttt{\textbar\space otherwise = False}\\ \hline
\end{tabularx}
\end{table}

SMURF allows 3 transformation operations to tone rows: inversion, retrograde and
transposition.
\begin{table} [H]
\centering
\begin{tabularx}{0.9\textwidth}{llX}
\hline\hline
Tone Row Operator & Description & Example \\
\hline\hline
   $\sim$ & Inversion & $\sim$ {\it row} (returns the inversion of {\it row})\\ \hline
   \textless\textgreater & Retrograde & \textless\textgreater~{\it row} (returns the
           retrograde of {\it row})\\ \hline
   $\wedge\wedge$ & Transposition & $\wedge\wedge$ {\it row} 3 (transpose {\it row} by 3
           semitones)\\ \hline
\end{tabularx}
\end{table}


\subsubsection{Whitespace}
Blanks, tabs and newlines are referred to as whitespaces in SMURF. SMURF treats
newlines as tokens and ignores others whitespaces. 


\subsection{Separators}
SMURF uses separators to seperate tokens. The separators of SMURF include:
\begin{verbatim} 
    ,   :   ;   {   }   whitespaces
\end{verbatim}

